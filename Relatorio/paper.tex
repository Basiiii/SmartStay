% This template has been tested with LLNCS DOCUMENT CLASS -- version 2.21 (12-Jan-2022)

% !TeX spellcheck = pt_PT-Portuguese
% !LTeX: language= pt_PT-Portuguese
% !TeX encoding = utf8
% !TeX program = pdflatex
% !BIB program = bibtex
% -*- coding:utf-8 mod:LaTeX -*-

% "a4paper" enables:
%
%  - easy print out on DIN A4 paper size
%
% One can configure a4 vs. letter in the LaTeX installation. So it is configuration dependend, what the paper size will be.
% This option  present, because the current word template offered by Springer is DIN A4.
% We accept that DIN A4 cause WTFs at persons not used to A4 in USA.
%
% "runningheads" enables:
%
%  - page number on page 2 onwards
%  - title/authors on even/odd pages
%
% This is good for other readers to enable proper archiving among other papers and pointing to
% content. Even if the title page states the title, when printed and stored in a folder, when
% blindly opening the folder, one could hit not the title page, but an arbitrary page. Therefore,
% it is good to have title printed on the pages, too.
%
% To disable outputting page headers and footers, remove "runningheads"
\documentclass[runningheads,a4paper,english]{llncs}[2022/01/12]

% backticks (`) are rendered as such in verbatim environments.
% See following links for details:
%   - https://tex.stackexchange.com/a/341057/9075
%   - https://tex.stackexchange.com/a/47451/9075
%   - https://tex.stackexchange.com/a/166791/9075
\usepackage{upquote}

% Used to add images
\usepackage{graphicx}

% Used for figures
\usepackage{float}

% Set English as language and allow to write hyphenated"=words
%
% Even though `american`, `english` and `USenglish` are synonyms for babel package (according to https://tex.stackexchange.com/questions/12775/babel-english-american-usenglish), the llncs document class is prepared to avoid the overriding of certain names (such as "Abstract." -> "Abstract" or "Fig." -> "Figure") when using `english`, but not when using the other 2.
% english has to go last to set it as default language
\usepackage[ngerman,main=english]{babel}
%
% Hint by http://tex.stackexchange.com/a/321066/9075 -> enable "= as dashes
\addto\extrasenglish{\languageshorthands{ngerman}\useshorthands{"}}
%
% Fix by https://tex.stackexchange.com/a/441701/9075
\usepackage{regexpatch}
\makeatletter
\edef\switcht@albion{%
  \relax\unexpanded\expandafter{\switcht@albion}%
}
\xpatchcmd*{\switcht@albion}{ \def}{\def}{}{}
\xpatchcmd{\switcht@albion}{\relax}{}{}{}
\edef\switcht@deutsch{%
  \relax\unexpanded\expandafter{\switcht@deutsch}%
}
\xpatchcmd*{\switcht@deutsch}{ \def}{\def}{}{}
\xpatchcmd{\switcht@deutsch}{\relax}{}{}{}
\edef\switcht@francais{%
  \relax\unexpanded\expandafter{\switcht@francais}%
}
\xpatchcmd*{\switcht@francais}{ \def}{\def}{}{}
\xpatchcmd{\switcht@francais}{\relax}{}{}{}
\makeatother

% Links behave as they should. Enables "\url{...}" for URL typesettings.
% Allow URL breaks also at a hyphen, even though it might be confusing: Is the "-" part of the address or just a hyphen?
% See https://tex.stackexchange.com/a/3034/9075.
\usepackage[hyphens]{url}

% When activated, use text font as url font, not the monospaced one.
% For all options see https://tex.stackexchange.com/a/261435/9075.
% \urlstyle{same}

% Improve wrapping of URLs - hint by http://tex.stackexchange.com/a/10419/9075
\makeatletter
\g@addto@macro{\UrlBreaks}{\UrlOrds}
\makeatother

% nicer // - solution by http://tex.stackexchange.com/a/98470/9075
% DO NOT ACTIVATE -> prevents line breaks
%\makeatletter
%\def\Url@twoslashes{\mathchar`\/\@ifnextchar/{\kern-.2em}{}}
%\g@addto@macro\UrlSpecials{\do\/{\Url@twoslashes}}
%\makeatother

% This is the modern package for "Computer Modern".
% In case this gets activated, one has to switch from cmap package to glyphtounicode (in the case of pdflatex)
\usepackage[%
    rm={oldstyle=false,proportional=true},%
    sf={oldstyle=false,proportional=true},%
    % By using 'variable=true' the monospaced font can be used as variable font (with differents widths per letter)
    % However, this makes listings look ugly.
    tt={oldstyle=false,proportional=true,variable=false},%
    qt=false%
]{cfr-lm}

% Has to be loaded AFTER any font packages. See https://tex.stackexchange.com/a/2869/9075.
\usepackage[T1]{fontenc}

% Character protrusion and font expansion. See http://www.ctan.org/tex-archive/macros/latex/contrib/microtype/

\usepackage[
  babel=true, % Enable language-specific kerning. Take language-settings from the languge of the current document (see Section 6 of microtype.pdf)
  expansion=alltext,
  protrusion=alltext-nott, % Ensure that at listings, there is no change at the margin of the listing
  final % Always enable microtype, even if in draft mode. This helps finding bad boxes quickly.
        % In the standard configuration, this template is always in the final mode, so this option only makes a difference if "pros" use the draft mode
]{microtype}

% \texttt{test -- test} keeps the "--" as "--" (and does not convert it to an en dash)
\DisableLigatures{encoding = T1, family = tt* }

%\DeclareMicrotypeSet*[tracking]{my}{ font = */*/*/sc/* }%
%\SetTracking{ encoding = *, shape = sc }{ 45 }
% Source: http://homepage.ruhr-uni-bochum.de/Georg.Verweyen/pakete.html
% Deactiviated, because does not look good

\usepackage{graphicx}

% Diagonal lines in a table - http://tex.stackexchange.com/questions/17745/diagonal-lines-in-table-cell
% Slashbox is not available in texlive (due to licensing) and also gives bad results. Thus, we use diagbox
\usepackage{diagbox}

\usepackage{xcolor}

% Code Listings
\usepackage{listings}
\usepackage{caption}

\definecolor{eclipseStrings}{RGB}{42,0.0,255}
\definecolor{eclipseKeywords}{RGB}{127,0,85}
\colorlet{numb}{magenta!60!black}

% JSON definition
% Source: https://tex.stackexchange.com/a/433961/9075

\lstdefinelanguage{json}{
    basicstyle=\normalfont\ttfamily,
    commentstyle=\color{eclipseStrings}, % style of comment
    stringstyle=\color{eclipseKeywords}, % style of strings
    numbers=left,
    numberstyle=\scriptsize,
    stepnumber=1,
    numbersep=8pt,
    showstringspaces=false,
    breaklines=true,
    frame=lines,
    % backgroundcolor=\color{gray}, %only if you like
    string=[s]{"}{"},
    comment=[l]{:\ "},
    morecomment=[l]{:"},
    literate=
        *{0}{{{\color{numb}0}}}{1}
         {1}{{{\color{numb}1}}}{1}
         {2}{{{\color{numb}2}}}{1}
         {3}{{{\color{numb}3}}}{1}
         {4}{{{\color{numb}4}}}{1}
         {5}{{{\color{numb}5}}}{1}
         {6}{{{\color{numb}6}}}{1}
         {7}{{{\color{numb}7}}}{1}
         {8}{{{\color{numb}8}}}{1}
         {9}{{{\color{numb}9}}}{1}
}

\lstset{
  % everything between (* *) is a latex command
  escapeinside={(*}{*)},
  %
  language=json,
  %
  showstringspaces=false,
  %
  extendedchars=true,
  %
  basicstyle=\footnotesize\ttfamily,
  %
  commentstyle=\slshape,
  %
  % default: \rmfamily
  stringstyle=\ttfamily,
  %
  breaklines=true,
  %
  breakatwhitespace=true,
  %
  % alternative: fixed
  columns=flexible,
  %
  numbers=left,
  %
  numberstyle=\tiny,
  %
  basewidth=.5em,
  %
  xleftmargin=.5cm,
  %
  % aboveskip=0mm,
  %
  % belowskip=0mm,
  %
  captionpos=b
}

% Enable Umlauts when using \lstinputputlisting.
% See https://stackoverflow.com/a/29260603/873282 für details.
% listingsutf8 did not work in June 2020.
\lstset{literate=
  {á}{{\'a}}1 {é}{{\'e}}1 {í}{{\'i}}1 {ó}{{\'o}}1 {ú}{{\'u}}1
  {Á}{{\'A}}1 {É}{{\'E}}1 {Í}{{\'I}}1 {Ó}{{\'O}}1 {Ú}{{\'U}}1
  {à}{{\`a}}1 {è}{{\`e}}1 {ì}{{\`i}}1 {ò}{{\`o}}1 {ù}{{\`u}}1
  {À}{{\`A}}1 {È}{{\'E}}1 {Ì}{{\`I}}1 {Ò}{{\`O}}1 {Ù}{{\`U}}1
  {ä}{{\"a}}1 {ë}{{\"e}}1 {ï}{{\"i}}1 {ö}{{\"o}}1 {ü}{{\"u}}1
  {Ä}{{\"A}}1 {Ë}{{\"E}}1 {Ï}{{\"I}}1 {Ö}{{\"O}}1 {Ü}{{\"U}}1
  {â}{{\^a}}1 {ê}{{\^e}}1 {î}{{\^i}}1 {ô}{{\^o}}1 {û}{{\^u}}1
  {Â}{{\^A}}1 {Ê}{{\^E}}1 {Î}{{\^I}}1 {Ô}{{\^O}}1 {Û}{{\^U}}1
  {Ã}{{\~A}}1 {ã}{{\~a}}1 {Õ}{{\~O}}1 {õ}{{\~o}}1
  {œ}{{\oe}}1 {Œ}{{\OE}}1 {æ}{{\ae}}1 {Æ}{{\AE}}1 {ß}{{\ss}}1
  {ű}{{\H{u}}}1 {Ű}{{\H{U}}}1 {ő}{{\H{o}}}1 {Ő}{{\H{O}}}1
  {ç}{{\c c}}1 {Ç}{{\c C}}1 {ø}{{\o}}1 {å}{{\r a}}1 {Å}{{\r A}}1
}

% For easy quotations: \enquote{text}
% This package is very smart when nesting is applied, otherwise textcmds (see below) provides a shorter command
\usepackage[autostyle=true]{csquotes}

% Enable using "`quote"' - see https://tex.stackexchange.com/a/150954/9075
\defineshorthand{"`}{\openautoquote}
\defineshorthand{"'}{\closeautoquote}

% Nicer tables (\toprule, \midrule, \bottomrule)
\usepackage{booktabs}

% Extended enumerate, such as \begin{compactenum}
\usepackage{paralist}

% Bibliopgraphy enhancements
%  - enable \cite[prenote][]{ref}
%  - enable \cite{ref1,ref2}
% Alternative: \usepackage{cite}, which enables \cite{ref1, ref2} only (otherwise: Error message: "White space in argument")

% Doc: http://texdoc.net/natbib
\usepackage[%
  square,        % for square brackets
  comma,         % use commas as separators
  numbers,       % for numerical citations;
  %sort           % orders multiple citations into the sequence in which they appear in the list of references;
  sort&compress  % as sort but in addition multiple numerical citations
                 % are compressed if possible (as 3-6, 15);
]{natbib}

% In the bibliography, references have to be formatted as 1., 2., ... not [1], [2], ...
\renewcommand{\bibnumfmt}[1]{#1.}

% Enable hyperlinked author names in the case of \citet
% Source: https://tex.stackexchange.com/a/76075/9075
\usepackage{etoolbox}
\makeatletter
\patchcmd{\NAT@test}{\else \NAT@nm}{\else \NAT@hyper@{\NAT@nm}}{}{}
\makeatother

% Prepare more space-saving rendering of the bibliography
% Source: https://tex.stackexchange.com/a/280936/9075
\SetExpansion
[ context = sloppy,
  stretch = 30,
  shrink = 60,
  step = 5 ]
{ encoding = {OT1,T1,TS1} }
{ }

% Put figures aside a text
\usepackage[rflt]{floatflt}

% Enable nice comments
\usepackage{pdfcomment}

\newcommand{\commentontext}[2]{\colorbox{yellow!60}{#1}\pdfcomment[color={0.234 0.867 0.211},hoffset=-6pt,voffset=10pt,opacity=0.5]{#2}}
\newcommand{\commentatside}[1]{\pdfcomment[color={0.045 0.278 0.643},icon=Note]{#1}}

% Compatibality with packages todo, easy-todo, todonotes
\newcommand{\todo}[1]{\commentatside{#1}}

% Compatiblity with package fixmetodonotes
\newcommand{\TODO}[1]{\commentatside{#1}}

% Put footnotes below floats
% Source: https://tex.stackexchange.com/a/32993/9075
\usepackage{stfloats}
\fnbelowfloat

\usepackage[group-minimum-digits=4,per-mode=fraction]{siunitx}

% Enable that parameters of \cref{}, \ref{}, \cite{}, ... are linked so that a reader can click on the number an jump to the target in the document
\usepackage{hyperref}

% Enable hyperref without colors and without bookmarks
\hypersetup{
  hidelinks,
  colorlinks=true,
  allcolors=black,
  pdfstartview=Fit,
  breaklinks=true
}

% Enable correct jumping to figures when referencing
\usepackage[all]{hypcap}

\usepackage[caption=false,font=footnotesize]{subfig}

\usepackage{mindflow}

% Extensions for references inside the document (\cref{fig:sample}, ...)
% Enable usage \cref{...} and \Cref{...} instead of \ref: Type of reference included in the link
% That means, "Figure 5" is a full link instead of just "5".
\usepackage[capitalise,nameinlink]{cleveref}

\crefname{section}{Sec.}{Sec.}
\Crefname{section}{Section}{Sections}
\crefname{listing}{List.}{List.}
\crefname{listing}{Listing}{Listings}
\Crefname{listing}{Listing}{Listings}
\crefname{lstlisting}{Listing}{Listings}
\Crefname{lstlisting}{Listing}{Listings}

\usepackage{lipsum}

% For demonstration purposes only
% These packages can be removed when all examples have been deleted
\usepackage[math]{blindtext}
\usepackage{mwe}
\usepackage[realmainfile]{currfile}
\usepackage{tcolorbox}
\tcbuselibrary{listings}

%introduce \powerset - hint by http://matheplanet.com/matheplanet/nuke/html/viewtopic.php?topic=136492&post_id=997377
\DeclareFontFamily{U}{MnSymbolC}{}
\DeclareSymbolFont{MnSyC}{U}{MnSymbolC}{m}{n}
\DeclareFontShape{U}{MnSymbolC}{m}{n}{
  <-6>    MnSymbolC5
  <6-7>   MnSymbolC6
  <7-8>   MnSymbolC7
  <8-9>   MnSymbolC8
  <9-10>  MnSymbolC9
  <10-12> MnSymbolC10
  <12->   MnSymbolC12%
}{}
\DeclareMathSymbol{\powerset}{\mathord}{MnSyC}{180}

\usepackage{xspace}
\newcommand{\eg}{e.g.,\ }
\newcommand{\ie}{i.e.,\ }

% Enable hyphenation at other places as the dash.
% Example: applicaiton\hydash specific
\makeatletter
\newcommand{\hydash}{\penalty\@M-\hskip\z@skip}
% Definition of "= taken from http://mirror.ctan.org/macros/latex/contrib/babel-contrib/german/ngermanb.dtx
\makeatother

% Add manual adapted hyphenation of English words
% See https://ctan.org/pkg/hyphenex and https://tex.stackexchange.com/a/22892/9075 for details
% Does not work on MiKTeX, therefore disabled - issue reported at https://github.com/MiKTeX/miktex-packaging/issues/271
% \input{ushyphex}

% correct bad hyphenation here
\hyphenation{op-tical net-works semi-conduc-tor}

% Add copyright
%
% This is recommended if you intend to send the version to colleagues
% See https://ctan.org/pkg/llncsconf for details
\iffalse
  % state: intended | submitted | llncs
  % you can add "crop" if the paper should be cropped to the format Springer is publishing
  \usepackage[intended]{llncsconf}

  \conference{name of the conference}

  % in case of "proceedings" (final version!)
  % example: \llncs{Anonymous et al. (eds). \emph{Proceedings of the International Conference on \LaTeX-Hacks}, LNCS~42. Some Publisher, 2016.}{0042}
  % 0042 denotes an example start page
  \llncs{book editors and title}{0042}
\fi

% Enable copy and paste of text from the PDF
% Only required for pdflatex. It "just works" in the case of lualatex.
% Alternative: cmap or mmap package
% mmap enables mathematical symbols, but does not work with the newtx font set
% See: https://tex.stackexchange.com/a/64457/9075
% Other solutions outlined at http://goemonx.blogspot.de/2012/01/pdflatex-ligaturen-und-copynpaste.html and http://tex.stackexchange.com/questions/4397/make-ligatures-in-linux-libertine-copyable-and-searchable
% Trouble shooting outlined at https://tex.stackexchange.com/a/100618/9075
%
% According to https://tex.stackexchange.com/q/451235/9075 this is the way to go
\input glyphtounicode
\pdfgentounicode=1

\begin{document}

\title{Gestão de Alojamentos Turísticos}
% If Title is too long, use \titlerunning
%\titlerunning{Short Title}
\subtitle{Programação Orientada a Objetos}

% Single insitute
\author{Enrique George Rodrigues Nº 28602}

% If there are too many authors, use \authorrunning
%\authorrunning{First Author et al.}

\institute{Instituto Politécnico do Cávado e do Ave\\
	Licenciatura Engenharia de Sistemas Informáticos\\
	\vspace{6pt}
	15 de Novembro de 2024}

%% Multiple insitutes - ALTERNATIVE to the above
% \author{%
%     Firstname Lastname\inst{1} \and
%     Firstname Lastname\inst{2}
% }
%
%If there are too many authors, use \authorrunning
%  \authorrunning{First Author et al.}
%
%  \institute{
%      Insitute 1\\
%      \email{...}\and
%      Insitute 2\\
%      \email{...}
%}

\maketitle

\renewcommand{\abstractname}{Resumo.} % Change the abstract title to "Resumo"
\begin{abstract}
	
Este projeto tem como objetivo o desenvolvimento de uma aplicação para a gestão de alojamentos turísticos, permitindo a administração eficiente de registos, reservas e consultas relacionadas com clientes e alojamentos. A solução foi implementada em C\#, utilizando o paradigma de programação orientada a objetos, de forma a garantir uma estrutura modular e escalável. As principais funcionalidades incluem o registo de clientes, a gestão de reservas, o processamento de check-ins e check-outs, bem como o controlo de pagamentos. A aplicação destaca-se pelo uso de dicionários como estruturas de dados eficientes, permitindo um acesso rápido e otimizado a registos de clientes, reservas e alojamentos, com uma complexidade temporal de $O(1)$ para operações de consulta e inserção. A persistência de dados é gerida em formato JSON e binária, proporcionando uma solução leve e prática para operadores turísticos.

\renewcommand{\abstractname}{Resumo}

% Redefine keywords manually since llncs doesn't define \keywordsname
\def\keywords{\vspace{.5em}
	{\textbf{Palavras-chave:}}\,\relax%
}

\keywords{registos, consultas, reservas, check-in, clientes, alojamentos}
\end{abstract}

\newpage
\setcounter{tocdepth}{3}
\renewcommand{\contentsname}{Índice}
\tableofcontents

%\listoffigures

% Ensure figures are numbered sequentially (without chapter number)
\renewcommand{\thefigure}{\arabic{figure}}
% Ensure listings are numbered sequentially (without chapter number)
\renewcommand{\thelstlisting}{\arabic{lstlisting}}
% Re-name listing names
\renewcommand\lstlistingname{Amostra de Código}
\renewcommand\lstlistlistingname{Amostras de Código}
\lstlistoflistings

\newpage
\setcounter{page}{1}
\section{Introdução}
\label{sec:introduction}

\subsection{Contexto e Motivação}
O setor do turismo enfrenta desafios crescentes na gestão eficiente de alojamentos, exigindo soluções que garantam agilidade, precisão e uma experiência satisfatória para os clientes. Este projeto visa desenvolver um sistema de gestão de alojamentos turísticos que permita a administração eficaz de registos de clientes, reservas e alojamentos. A implementação foi realizada em C\#, uma linguagem reconhecida pela sua robustez e flexibilidade, permitindo o desenvolvimento de aplicações escaláveis e eficientes.

\subsection{Estrutura de Dados e Persistência}  
Para garantir um desempenho optimizado, o sistema utiliza dicionários como estruturas de dados principais, baseados em tabelas de hash, que proporcionam um acesso rápido e eficiente às informações cruciais, com uma complexidade temporal de $O(1)$ para operações de consulta e inserção.  

Adicionalmente, a persistência de dados é gerida em formato JSON, o que facilita a integração com outros sistemas e a manipulação dos dados de forma legível e estruturada. No entanto, para operações que exigem maior eficiência em termos de tempo de leitura/escrita, como guardar ou carregar rapidamente os dados sem necessidade de exportação ou importação, utilizei o \textbf{protobuf-net} para serializar e desserializar os dados em formato binário. Esta abordagem combina a flexibilidade do JSON com a velocidade do formato binário, garantindo um equilíbrio entre interoperabilidade e desempenho.  

\subsection{Boas Práticas de Programação}
Embora a eficiência e a funcionalidade sejam os pilares deste projeto, a aplicação também foi criada com atenção às boas práticas de programação, promovendo um código limpo e manutenível. Desta forma, o sistema não apenas atende às necessidades atuais do setor, mas também está preparado para futuras evoluções e inovações na gestão de alojamentos turísticos.

\subsection{Objetivos e Organização do Documento}
Este documento descreve o trabalho prático da unidade curricular de Programação Orientada a Objetos, parte integrante da Licenciatura em Engenharia de Sistemas Informáticos no Instituto Politécnico do Cávado e do Ave. O enunciado deste trabalho prático pode ser encontrado no anexo titulado ``Trabalho\_POO\_ESI\_2024\_2025``. O código fonte está disponível no Github e pode ser consultado \href{https://github.com/Basiiii/SmartStay}{aqui}.

\newpage
\section{Padrões e Práticas de Programação}
\label{sec:codingstyle}

Para garantir a consistência, legibilidade e manutenção da base de código, este projeto segue os padrões de programação estabelecidos. Estes padrões estão alinhados com o Guia de Estilo da Microsoft para C\# \cite{microsoftcsharpstyle}, promovendo uniformidade em todo o projeto.

\subsection{Convenções de Nomeação}

A adoção de uma convenção de nomeação clara melhora a clareza do código. As seguintes convenções foram utilizadas ao longo do projeto:

\begin{itemize}
	\item \textbf{Classes e Structs}: \texttt{ClassName}
	\item \textbf{Métodos}: \texttt{MethodName()}
	\item \textbf{Propriedades}: \texttt{PropertyName}
	\item \textbf{Variáveis de Instância e Locais}: \texttt{variableName}
	\item \textbf{Variáveis privadas}: \texttt{\_variableName}
	\item \textbf{Constantes}: \texttt{CONSTANT\_NAME}
	\item \textbf{Parâmetros}: \texttt{parameterName}
	\item \textbf{Enums}: \texttt{EnumName}
	\item \textbf{Métodos de Teste}: Seguem o padrão \texttt{MethodName\_Condition\_ExpectedOutcome}
\end{itemize}

Evita-se o uso de abreviações excessivas e nomes confusos, promovendo a clareza no código.

\subsection{Regras de Formatação}

A formatação consistente é fundamental para a legibilidade:

\begin{itemize}
	\item \textbf{Indentação}: 4 espaços por nível de indentação, evitando tabs.
	\item \textbf{Comprimento da Linha}: Linhas limitadas a um máximo de 120 caracteres.
	\item \textbf{Chaves}: As chaves de abertura são colocadas na próxima linha da declaração de controle.
	\item \textbf{Espaçamento}: Espaços em branco são usados para separar operadores e ao redor de chaves.
	\item \textbf{Comentários}: Comentários descritivos são utilizados para explicar seções complexas do código, seguindo um estilo consistente.
\end{itemize}

\subsection{Práticas de Programação}

Seguir as melhores práticas garante a qualidade do código:

\begin{itemize}
	\item Utilizam-se nomes claros para variáveis e métodos, tornando o código autoexplicativo.
	\item Escreve-se código claro e conciso, evitando complexidade desnecessária.
	\item Segue-se o Guia de Estilo da Microsoft para C\# \cite{microsoftcsharpstyle} para formatação e estilo gerais, garantindo a conformidade com os padrões da indústria.
\end{itemize}

\subsection{Controlo de Versões}

Utiliza-se Git para o controle de versões, facilitando a gestão eficaz da base de código:

\begin{itemize}
	\item Cria-se uma nova branch para cada funcionalidade ou correção de erro.
	\item As branches são criadas a partir da branch principal para desenvolvimento, com nomes descritivos que refletem a funcionalidade ou erro abordado.
	\item Evita-se fazer commits diretamente na branch principal, garantindo a estabilidade.
	\item As branches de funcionalidade são regularmente fundidas de volta na branch principal após testes completos.
\end{itemize}

\subsection{Documentação}

Utiliza-se \textit{XML Documentation Comments} para documentar o código, permitindo a geração automática de documentação da API:

\begin{itemize}
	\item Segue-se a sintaxe do XML para comentários, documentando métodos, propriedades e classes.
	\item A documentação inclui descrições breves, descrições de parâmetros, descrições de valores de retorno e exemplos de uso.
	\item Utilizam-se tags como \texttt{<summary>}, \texttt{<param>}, \texttt{<returns>} e \texttt{<example>} para estruturar corretamente os comentários.
	\item Define-se o autor nos comentários de cabeçalho de classes para novos ficheiros ou seções significativas de código.
\end{itemize}

\subsection{Padrões de Design Utilizados}

O sistema implementado segue princípios sólidos de engenharia de software, recorrendo a padrões de design bem estabelecidos para garantir uma arquitectura modular, reutilizável e fácil de manter. Dois dos principais padrões utilizados foram:

\begin{itemize}
	\item \textbf{MVC (Model-View-Controller)}: Este padrão foi utilizado para separar as preocupações do sistema. O backend implementa as regras de negócio e a manipulação de dados no modelo (Model), enquanto o frontend é responsável pela apresentação visual (View) e interage com o modelo através de controladores (Controllers). Este design melhora a organização do código e facilita a introdução de alterações numa camada sem impactar as outras.
	
	\item \textbf{Facade (Face)}: Uma classe face do sistema foi utilizada para simplificar e abstrair interações complexas entre diferentes componentes do sistema. Este padrão oferece uma interface simples e clara para funcionalidades comuns.
\end{itemize}

Estes padrões permitiram a divisão eficiente do código em duas camadas principais: backend e frontend, promovendo um design coeso e robusto.

\subsection{Funções Lambda}

O sistema também aproveita as funções lambda para simplificar a manipulação de coleções e propriedades. Estas expressões são frequentemente utilizadas para criar acessores e transformar coleções de forma concisa e eficiente. Seguem dois exemplos ilustrativos:

\vspace*{15pt}
\begin{lstlisting}[ 
	language=csh, 
	caption={Getter e Setter com Função Lambda}, 
	label={lst:lambdaGetterSetter}]
public static int LastAssignedId
{
	get => _lastAccommodationId;
	set => _lastAccommodationId = value;
}
\end{lstlisting}

No exemplo acima, as expressões lambda proporcionam uma forma compacta e clara de definir os métodos de acesso para propriedades.

\newpage
\section{Diagrama de Classes}
\label{sec:classes}

\subsection{Introdução ao Sistema}
O diagrama de classes descreve a estrutura lógica do sistema de gestão de reservas para alojamentos turísticos, com uma arquitetura desenhada para centralizar e organizar informações essenciais sobre clientes, proprietários, reservas, alojamentos e pagamentos. Este modelo, por além de otimizar o fluxo de trabalho, promove a extensibilidade e modularidade, facilitando tanto as expansões futuras como a manutenção do sistema.

\subsection{Classes Principais}

No diagrama, cada classe representa uma componente fundamental do sistema. A classe \textit{Client} (cliente) guarda informações sobre um cliente individual, como nome, e-mail, telefone e endereço. A classe \textit{Clients} funciona como uma coleção que organiza várias instâncias de \textit{Client} utilizando um dicionário, oferecendo acesso eficiente e métodos para manipulação como \texttt{Add()} e \texttt{Remove()}.

De forma semelhante, a classe \textit{Owner} (proprietário) guarda dados sobre os proprietários, incluindo informações pessoais e as acomodações associadas. A classe \textit{Owners} organiza múltiplos \textit{Owner} utilizando um dicionário, facilitando operações como adição, remoção e pesquisa.

A classe \textit{Accommodation} (alojamento) representa as unidades disponíveis para reserva. Cada alojamento inclui detalhes como o tipo de alojamento, o nome e a lista de quartos associados (\textit{Rooms}). A classe \textit{Room} fornece detalhes específicos sobre os quartos, como tipo, preço por noite e disponibilidade, além de funcionalidades para calcular custos e verificar disponibilidade. Para gerir vários alojamentos, a classe \textit{Accommodations} organiza instâncias de \textit{Accommodation} utilizando um dicionário, com métodos como \texttt{FindRoomById()} e \texttt{RemoveRoom()}.

A classe \textit{Reservation} (reserva) liga clientes a acomodações, registrando detalhes como datas de check-in e check-out, tipo de alojamento, custo total e estado da reserva. Ela também mantém uma lista de pagamentos (\textit{Payment}) associados, permitindo verificar se os pagamentos estão atualizados. A classe \textit{Reservations} armazena várias reservas em um dicionário e oferece funcionalidades para adicionar, cancelar ou pesquisar reservas.

A classe \textit{Payment} (pagamento) é responsável por registar informações detalhadas sobre transações, incluindo o montante pago, método de pagamento e data.

\subsection{Processamento de Pagamentos} A classe \textit{Payment} lida com detalhes individuais das transações financeiras associadas a reservas. A classe \textit{Reservation} mantém uma lista de pagamentos e fornece métodos como \texttt{MakePayment()} para processar pagamentos e verificar o estado financeiro.

\subsection{BookingManager e a Interface ManageableEntity} A classe \textit{BookingManager} atua como o controlador central/face do sistema, sendo responsável por gerir os clientes, proprietários, reservas e alojamentos. Ela utiliza instâncias das classes \textit{Clients}, \textit{Owners}, \textit{Reservations} e \textit{Accommodations}, centralizando as operações.

Para uniformizar as operações de gestão, o sistema utiliza a interface \textit{ManageableEntity}, implementada pelas classes de coleção (\textit{Clients}, \textit{Owners}, \textit{Reservations} e \textit{Accommodations}). Esta interface define métodos comuns, como \textit{Add()}, \textit{Remove()}, \textit{Export()} e \textit{Load()}.

O \textit{BookingManager} centraliza operações comuns, como \texttt{CreateBasicClient()}, \texttt{FindClientById()} e \texttt{SaveAll()}, fornecendo uma interface unificada e simples para os utilizadores.

\subsection{Padrão de Face (Facade Pattern)} 
O \textit{BookingManager} implementa o \textit{Padrão Facade} \cite{facadepattern}, funcionando como um ponto de acesso único que simplifica a interação com o sistema. Em vez de chamar métodos individuais nas classes de coleção, o \textit{BookingManager} oferece métodos como \texttt{CreateCompleteOwner()}, \texttt{UpdateReservation()} e \texttt{CancelReservation()}, abstraindo a complexidade interna.

Este padrão promove uma maior modularidade e escalabilidade do sistema, permitindo a introdução de novas funcionalidades sem impactar a interface externa.

\begin{figure}[h!]
	\centering
	\includegraphics[width=0.97\textwidth]{Images/POO_Class_Diagram.png}
	\caption{Diagrama de Classes do Sistema de Gestão de Alojamentos Turísticos}
	\label{fig:diagrama_classes}
\end{figure}

\newpage
\section{Organização do Código}
\label{sec:organizacaocodigo}

A arquitetura do sistema foi estruturada de acordo com um layout comum e amplamente utilizado em aplicações de software, que segue o padrão de separação de responsabilidades entre diferentes camadas. Este padrão facilita a manutenção, escalabilidade e testabilidade do sistema. A divisão em \textit{Models}, \textit{Repositories}, \textit{Services} e \textit{Utilities} organiza a aplicação de maneira modular e coesa, garantindo que cada componente tenha uma função clara e bem definida. Cada classe é guardada num ficheiro chamado de forma correspondente, reforçando a organização e facilitando a navegação no código.

\subsection{Nomeação de Ficheiros}
Para manter uma estrutura organizada e intuitiva, os ficheiros do sistema foram nomeados de acordo com as classes que eles contêm. Cada classe tem um ficheiro correspondente com o mesmo nome, garantindo uma correspondência direta entre a funcionalidade e o ficheiro onde ela está implementada. Esta abordagem facilita a localização de classes específicas e melhora a navegabilidade do projeto, uma vez que o nome de cada ficheiro reflete o conteúdo e a responsabilidade do mesmo. Por exemplo, a classe \textit{Client} está localizada no ficheiro \texttt{Client.cs}, enquanto a classe \textit{BookingManager} pode ser encontrada em \texttt{BookingManager.cs}.

\subsection{Conformidade com a CLS}
A \textit{Common Language Specification (CLS)} define um conjunto de regras e padrões para garantir a interoperabilidade entre as diferentes linguagens que compõem o ecossistema .NET. Ao desenvolver o sistema, foi dada atenção à conformidade com a CLS para assegurar que o código seja acessível e utilizável por outras linguagens compatíveis com o .NET.

Dentro deste contexto, o tipo \texttt{decimal} foi escolhido para operações financeiras e monetárias devido à sua alta precisão e, também, por ser um tipo que segue as normas da CLS. Esta conformidade permite que o sistema mantenha uma representação exata de valores monetários e seja facilmente reutilizável em outros ambientes e linguagens que suportem .NET. 

Cabe destacar que nem todos os tipos numéricos são compatíveis com a CLS — por exemplo, tipos inteiros não-assinados (\textit{unsigned}) como \texttt{uint} e \texttt{ulong} não são \textit{CLS-compliant} e podem apresentar problemas de interoperabilidade em algumas linguagens. A escolha de tipos compatíveis com a CLS reforça o compromisso com a robustez e a acessibilidade do sistema em um ambiente .NET multi-linguagem.

\subsection{Models}
A camada de \textit{Models} é responsável por representar os dados essenciais do sistema. Nesta camada, as classes de modelo (como \textit{Owner}, \textit{Client}, \textit{Accommodation} e \textit{Reservation}) são definidas com propriedades e comportamentos relacionados aos dados que representam. Estes modelos são utilizados frequentemente para mapear as entidades que serão manipuladas pela aplicação, sendo a base para as operações de armazenamento de dados.

\subsection{Repositories}
A camada de \textit{Repositories} no nosso sistema é responsável por guardar e organizar listas de objetos de diferentes tipos, como Proprietários, Clientes, Alojamentos e Reservas. Esta camada serve como um \textit{container} centralizado para gerir estes conjuntos de dados durante a execução do programa, proporcionando um acesso estruturado e consistente às informações.

\subsection{Services}
A camada de \textit{Services} contém a lógica de negócio principal da aplicação. Os serviços orquestram as interações entre os modelos e os repositórios, executando operações mais complexas que envolvem várias entidades ou ações. No contexto do nosso sistema de gestão de reservas, o \textit{BookingManager} pode ser considerado um serviço, pois gere todos os processos complexos de dados.

\subsection{Utilities}
A camada de \textit{Utilities} contém classes e métodos auxiliares que oferecem funcionalidades que podem ser usadas em diferentes partes do sistema. As \textit{Utilities} ajudam a manter o código limpo e reutilizável, evitando a duplicação de lógica em várias partes da aplicação.

\subsection{Bibliotecas}  
A organização do código também foi complementada com a utilização de bibliotecas específicas, guardadas na pasta \texttt{Libraries}, para promover a separação de responsabilidades e a modularidade do sistema. Estas bibliotecas foram estruturadas para encapsular diferentes funcionalidades do sistema, permitindo uma gestão mais clara e eficiente das preocupações. Abaixo descrevemos as principais bibliotecas criadas:  

\begin{itemize}  
	\item \textbf{SmartStay.Common}: Esta biblioteca foi criada para conter elementos comuns que são utilizados por todas as partes do sistema. Ela inclui definições de \textit{enums}, \textit{exceptions} e outros componentes genéricos que podem ser reutilizados de maneira transversal. Ao centralizar esses elementos na biblioteca \texttt{SmartStay.Common}, é possível reduzir a redundância de código e melhorar a consistência em diferentes partes da aplicação.
	
	\item \textbf{SmartStay.Core}: A biblioteca central do sistema, \texttt{SmartStay.Core}, agrupa os modelos, repositórios, serviços e utilitários. Esta biblioteca encapsula a lógica de negócio principal e os elementos estruturais do sistema, assegurando uma clara separação entre os dados e as operações que os manipulam. Por exemplo, todas as classes de modelo, como \texttt{Client} e \texttt{Reservation}, bem como os repositórios que armazenam listas de objetos, estão organizados nesta biblioteca.
	
	\item \textbf{SmartStay.IO}: A biblioteca \texttt{SmartStay.IO} é responsável por lidar com operações de entrada e saída, particularmente o gerenciamento de dados em formato \textit{JSON}. Ela encapsula toda a lógica relacionada à leitura e escrita de informações, permitindo que o sistema realize a persistência de dados de maneira modular. Esta abordagem melhora a manutenção, pois mudanças na forma como os dados são armazenados ou carregados podem ser feitas exclusivamente nesta biblioteca, sem impactar outras partes do sistema.
	
	\item \textbf{SmartStay.Validation}: A biblioteca \texttt{SmartStay.Validation} foi projetada para centralizar toda a lógica de validação de dados do sistema. Ela inclui regras específicas para garantir a integridade dos dados inseridos e manipulados pelo sistema, como validações de datas e formatos de entrada. Este design promove a reutilização de lógica de validação em diferentes componentes, aumentando a consistência e facilitando futuras alterações ou expansões nas regras de validação.
\end{itemize}  

A utilização destas bibliotecas reflete um compromisso com a separação de preocupações (\textit{Separation of Concerns}), promovendo uma arquitetura limpa, extensível e de fácil manutenção. Cada biblioteca tem um propósito bem definido e pode ser atualizada ou substituída de forma independente, reduzindo a complexidade e os riscos associados a alterações no sistema.  

\newpage
\section{Arquitetura em Camadas (Layered Architecture)}
\label{sec:arquiteturacamadas}

A adoção de uma arquitetura em camadas é uma prática muito reconhecida para organizar e estruturar o código de uma aplicação, visando separar responsabilidades e promover escalabilidade e manutenção. Dentro desta abordagem, o padrão de \textbf{Model-View-Controller (MVC)} \cite{mvc} foi escolhido para gerir a interface de utilizador e a lógica de aplicação. Este padrão permite uma divisão clara entre as camadas que lidam com os dados, a lógica de negócio e a apresentação visual, facilitando o desenvolvimento modular e a evolução do sistema ao longo do tempo.

\subsection{Camadas da Arquitetura}

A arquitetura do sistema é composta pelas seguintes camadas principais:

\begin{itemize}
	 \item \textbf{Model}: Representa os dados do sistema e encapsula a lógica de negócios principal. 
	 \item \textbf{View}: Responsável pela apresentação visual e interação com o utilizador. 
	 \item \textbf{Controller}: Atua como intermediário entre a interface de utilizador e a lógica de negócios, processando interações e executando as operações necessárias.
	 \item \textbf{Repository e Service}: Fornecem suporte adicional à lógica do \textit{Model}, gerindo persistência de dados e operações de negócios específicas. 
\end{itemize}

Esta estrutura modular assegura que cada componente tem um papel claramente definido, facilitando a manutenção e evolução do sistema.

\subsection{Escolha do Padrão MVC}

O padrão MVC foi escolhido pelas suas características de modularidade e flexibilidade, ideais para sistemas com alta interação entre utilizadores, dados e regras de negócio. A separação entre Model, View e Controller garante que as alterações em uma camada não impactam diretamente as demais, promovendo robustez no desenvolvimento.

As vantagens do MVC incluem:

\begin{itemize} 
	\item \textbf{Facilidade de manutenção}: Cada camada pode ser atualizada ou substituída independentemente. 
	\item \textbf{Escalabilidade}: O sistema pode ser expandido de forma simples. 
	\item \textbf{Testabilidade}: As regras de negócio e os dados podem ser testados separadamente da interface de utilizador. 
\end{itemize}

\subsection{Desenvolvimento Progressivo}

O desenvolvimento foi conduzido de forma progressiva, com as camadas das regras de negócios implementadas antes da camada de apresentação:

\begin{enumerate} 
	\item \textbf{Camadas Fundamentais}: O desenvolvimento inicial focou-se em \textit{Model}, \textit{Repository}, \textit{Service}, \textit{Utility} e \textit{Validation}. Estas camadas foram projetadas, implementadas e testadas de forma independente, garantindo uma base sólida para o sistema. \item \textbf{Camada de Interface (View)}: Após a conclusão das camadas fundamentais, a interface de utilizador foi integrada, utilizando a lógica já validada para proporcionar uma experiência intuitiva aos utilizadores.	
\end{enumerate}

Este fluxo de desenvolvimento garantiu que as regras de negócio estivessem bem estruturadas e funcionais antes da criação da interface, permitindo uma evolução coesa do sistema.

\newpage
\section{Estruturas de Dados Utilizadas}
\label{sec:estruturasdados}

\subsection{Utilização de Dictionary}
As estruturas de dados escolhidas para o sistema foram definidas com o objetivo de otimizar a eficiência e o desempenho, especialmente em operações de busca, inserção e remoção. A principal estrutura de dados utilizada foi o \textit{Dictionary} (dicionário), que oferece uma maneira eficiente de armazenar e aceder a entidades como clientes, reservas e alojamentos, por meio de chaves únicas. Esta estrutura garante um tempo de acesso quase constante para operações como inserção e busca, o que é essencial para o bom desempenho do sistema, especialmente à medida que o número de entidades cresce.

\vspace*{15pt}
\begin{lstlisting}[ 
	language=csh, 
	caption={Uso de Dictionary}, 
	label={lst:C}]
readonly Dictionary<int, Client> _clientDictionary = new Dictionary<int, Client>();
\end{lstlisting}

O uso do \textit{Dictionary} garante que as operações de inserção e remoção de clientes sejam realizadas de forma rápida, com um tempo médio de execução constante $O(1)$, o que melhora a escalabilidade do sistema à medida que mais clientes são adicionados.

\subsection{Utilização de SortedSet}

Para a classe \textit{Room}, foi utilizada a estrutura \textit{SortedSet<DateRange>} para guardar os intervalos de datas das reservas. O \textit{SortedSet} é uma coleção que mantém os elementos em ordem, o que permite realizar buscas e consultas de maneira eficiente. A principal vantagem do \textit{SortedSet} sobre outras estruturas, como listas ou dicionários, é que ele mantém a ordem dos elementos automaticamente e proporciona operações eficientes de inserção, remoção e busca.

O \textit{SortedSet} é baseado numa árvore binária balanceada, tipicamente uma árvore vermelho-preta (\textit{Red-Black Tree}), que garante um tempo de execução logarítmico para as operações básicas: inserção, remoção e busca, ou seja, $O(log n)$. Isto acontece porque, ao usar uma árvore binária balanceada, o \textit{SortedSet} mantém o conjunto de dados ordenados, sem a necessidade de reordenar os elementos manualmente a cada operação.

Abaixo está um exemplo de como o sistema verifica a disponibilidade de um alojamento utilizando o \textit{SortedSet<DateRange>}:

\vspace*{15pt}
\begin{lstlisting}[ 
	language=csh, 
	caption={Verificação da Disponibilidade de um Alojamento}, 
	label={lst:C}]
public bool IsAvailable(DateTime startDate, DateTime endDate, DateRange? existingReservationRange = null)
{
	if (endDate <= startDate)
	throw new ArgumentException("End date must be after the start date.");
	
	var newReservation = new DateRange(startDate, endDate);
	
	// Get potential conflicting reservations within the requested range
	var potentialConflicts = _reservationDates.GetViewBetween(
	new DateRange(DateTime.MinValue, startDate), // All reservations that end before the start
	new DateRange(DateTime.MaxValue, endDate)    // All reservations that start after the end
	);
	
	// Check if there are any overlapping reservations
	foreach (var existingReservation in potentialConflicts)
	{
		// Skip the existing reservation if it's the one we're trying to modify
		if (existingReservation.Equals(existingReservationRange))
		{
			continue; // Skip this reservation as it's the one we're modifying
		}
		
		// An overlap occurs if the start date is before the end date, and the end date is after the start date
		if ((newReservation.Start < existingReservation.End) && (newReservation.End > existingReservation.Start))
		{
			return false; // There's an overlap, so the accommodation is not available
		}
	}
	
	return true; // No overlap found, accommodation is available
}
\end{lstlisting}

O uso do \textit{SortedSet} permite que o sistema verifique rapidamente se há conflitos de datas para uma nova reserva. Quando uma nova reserva é solicitada, o método \texttt{GetViewBetween} é utilizado para obter apenas as reservas que podem potencialmente se sobrepor com a nova reserva, o que torna o processo de verificação muito mais eficiente.

A principal vantagem do \textit{SortedSet} é que ele mantém os dados ordenados automaticamente, e as operações de busca e inserção são realizadas em tempo $O(log n)$, o que é muito mais eficiente do que um algoritmo de busca linear que teria um custo de $O(n)$. Em sistemas com muitas reservas, isto resulta num ganho de desempenho significativo.

Além disso, a estrutura \textit{SortedSet} oferece uma forma eficiente de lidar com a inserção de novos intervalos de datas, pois a árvore binária balanceada garante que as operações de inserção e remoção também sejam realizadas de maneira logarítmica. Isso é fundamental para garantir que o sistema permaneça escalável à medida que o número de reservas aumenta.

\subsection{Vantagens das Estruturas de Dados Escolhidas}

As escolhas do \textit{Dictionary} e \textit{SortedSet} refletem uma análise cuidadosa das necessidades do sistema, como a eficiência de acesso, a escalabilidade e a organização dos dados. Abaixo, destacam-se as principais vantagens dessas escolhas:

\begin{itemize}
	\item \textbf{Eficiência de Acesso}: O \textit{Dictionary} permite acesso rápido às entidades, como clientes e reservas, através de chaves únicas. Esta característica é fundamental para garantir o bom desempenho do sistema, especialmente quando precisa de lidar com grandes volumes de dados.
	\item \textbf{Escalabilidade}: O uso do \textit{Dictionary} para guardar entidades e do \textit{SortedSet} para as datas das reservas garante um sistema escalável. À medida que o número de clientes ou reservas aumenta, o desempenho mantém-se estável devido à eficiência destas estruturas.
	\item \textbf{Verificação Rápida de Conflitos de Datas}: O \textit{SortedSet} facilita a verificação de conflitos de datas de reservas, graças à sua capacidade de manter os dados ordenados automaticamente. A combinação com a operação \texttt{GetViewBetween} reduz significativamente o tempo de busca e garante uma execução eficiente mesmo com um número elevado de reservas.
\end{itemize}

Estas escolhas refletem uma análise cuidadosa dos requisitos de desempenho e eficiência do sistema, buscando sempre o equilíbrio entre a simplicidade na implementação o alto desempenho. Isto garante um sistema não apenas eficaz, mas também fácil de manter e escalar ao longo do tempo.

\newpage
\section{Uso de LINQ}
\label{sec:usodelinq}

No desenvolvimento do sistema, foi utilizado o LINQ (\textit{Language Integrated Query}) para facilitar a consulta e manipulação de coleções de dados, proporcionando uma forma mais concisa e legível de trabalhar com dados em memória. O LINQ integra-se diretamente no código C\#, permitindo escrever consultas diretamente nas coleções, sem necessidade de recorrer a linguagens de consulta externas ou a loops complicados.

\subsection{Vantagens de LINQ}

O LINQ oferece várias vantagens, entre as quais se destacam:

\begin{itemize}
	\item \textbf{Sintaxe Concisa e Legível}: Permite escrever consultas de forma mais direta e legível, o que resulta em código mais limpo e fácil de entender.
	\item \textbf{Integração com Coleções de Dados}: Pode ser utilizado diretamente sobre coleções de objetos como listas, arrays, e dicionários, o que facilita a manipulação de dados sem recorrer a ferramentas externas.
	\item \textbf{Redução de Erros e Complexidade}: Ao abstrair a complexidade da consulta e manipulação de dados, o LINQ reduz as possibilidades de erros e torna o código mais modular e menos propenso a falhas.
	\item \textbf{Execução Diferida}: O LINQ utiliza execução diferida, ou seja, a consulta só é executada quando necessário, o que pode melhorar o desempenho, especialmente quando se lida com um volume grande de dados.
\end{itemize}

\subsection{Utilização de LINQ no Código}

A seguir, são apresentados exemplos de como o LINQ foi utilizado no código para realizar operações de filtragem em coleções de dados.

O primeiro exemplo mostra como o LINQ é usado para encontrar todas as reservas associadas a um cliente, dado o seu ID único. A função `\textit{FindReservationsByClientId()}` filtra o dicionário de reservas, retornando apenas as reservas que correspondem ao `\textit{clientId}` fornecido.

\vspace*{15pt}
\renewcommand{\lstlistingname}{Amostra de Código}
\begin{lstlisting}[ 
	language=csh, 
	caption={Encontra Reservas por ID de Cliente}, 
	label={lst:C}],
public IEnumerable<Reservation> FindReservationsByClientId(int clientId)
{
	return _reservationDictionary.Values.Where(r => r.ClientId == clientId);
}
\end{lstlisting}

No segundo exemplo, o LINQ é utilizado para encontrar todas as reservas associadas a um alojamento específico, filtrando as reservas pelo `AccommodationId`. A função `FindReservationsByAccommodationId` realiza uma consulta parecida, mas com base no ID do alojamento.

\begin{lstlisting}[ 
	language=csh, 
	caption={Encontra Reservas por ID de Alojamento}, 
	label={lst:C}]
public IEnumerable<Reservation> FindReservationsByAccommodationId(int accommodationId)
{
	return _reservationDictionary.Values.Where(r => r.AccommodationId == accommodationId);
}
\end{lstlisting}

Estes exemplos demonstram como o LINQ simplifica a filtragem de dados nas coleções, tornando o código mais legível e eficiente.

\newpage
\section{Tratamento de Exceções}
\label{sec:tratamentoexcecoes}

A escolha de utilizar códigos de erro, ao invés de mensagens \textit{hardcoded}, proporciona maior flexibilidade, reutilização e facilita a manutenção do código. Além disso, permite uma melhor adaptação a mudanças nas regras de negócio, sem a necessidade de alteração direta nas mensagens de erro.

\subsection{Estrutura das Exceções}
A base para o tratamento de exceções no sistema é a classe \texttt{ValidationException}. Esta classe personalizada herda da classe base \texttt{Exception} e é usada para encapsular falhas de validação que ocorrem durante o processamento dos dados.

A principal característica da \texttt{ValidationException} é que ela recebe um código de erro (do tipo \texttt{ValidationErrorCode}) e, ao ser instanciada, busca automaticamente a mensagem correspondente através da classe \texttt{ValidationErrorMessages}. O código abaixo apresenta a implementação básica desta exceção personalizada:

\begin{lstlisting}[
	language=csh, 
	caption={Classe ValidationException}, 
	label={lst:C}]
public class ValidationException : Exception
{
	public ValidationErrorCode ErrorCode { get; }
	
	public ValidationException(ValidationErrorCode errorCode) : base(ValidationErrorMessages.GetErrorMessage(errorCode))
	{
		ErrorCode = errorCode;
	}
}
\end{lstlisting}

Como pode ser visto, ao receber um \texttt{ValidationErrorCode}, a classe \texttt{ValidationException} utiliza o método \texttt{GetErrorMessage} da classe \texttt{ValidationErrorMessages} para obter a mensagem correspondente ao erro, evitando a necessidade de codificar as mensagens diretamente na aplicação.

\subsection{Definição dos Códigos de Erro}
A enumeração \texttt{ValidationErrorCode} é responsável por definir os diferentes tipos de erros que podem ocorrer durante a validação dos dados. Cada valor desta enumeração é associado a um código numérico único, o que permite identificar de forma eficiente qual validação falhou.

Abaixo está um exemplo simplificado desta enumeração:

\begin{lstlisting}[
	language=csh, 
	caption={Enumeração ValidationErrorCode}, 
	label={lst:C}]
namespace SmartStay.Validation
{
	public enum ValidationErrorCode
	{
		InvalidName = 1001,
		InvalidEmail = 1002,
		InvalidPhoneNumber = 1003
	}
}
\end{lstlisting}

\subsection{Mensagens de Erro Dinâmicas}
A classe \texttt{ValidationErrorMessages} é responsável por buscar as mensagens de erro localizadas a partir de ficheiros de recursos. Em vez de guardar as mensagens diretamente no código, as mensagens são mantidas em ficheiros de recursos (.resx), o que facilita a tradução e manutenção das mensagens de erro, especialmente em sistemas com muitos idiomas.

Veja abaixo um exemplo simplificado da classe \textit{ValidationErrorMessages}:

\begin{lstlisting}[
	language=csh, 
	caption={Classe ValidationErrorMessages}, 
	label={lst:C}]
public static class ValidationErrorMessages
{
	private static readonly ResourceManager _resourceManager = new ResourceManager(
	"SmartStay.Validation.Resources.ValidationMessages", typeof(ValidationErrorMessages).Assembly);
	
	public static string GetErrorMessage(ValidationErrorCode errorCode)
	{
		return _resourceManager.GetString(errorCode.ToString(), CultureInfo.CurrentCulture) ??
		"Unknown validation error.";
	}
}
\end{lstlisting}

Desta forma, quando uma exceção de validação é gerada, a mensagem correspondente ao código de erro é obtida dinamicamente a partir do ficheiro de recursos com base na cultura atual. Caso o código de erro não esteja mapeado, uma mensagem padrão ''Unknown validation error.'' \textit{(Erro de validação desconhecido)} será devolvida, garantindo que todas as exceções são tratadas de forma consistente e traduzidas corretamente para o idioma apropriado.

\subsection{Benefícios da Abordagem}
A utilização de códigos de erro e mensagens dinâmicas traz uma série de benefícios para o sistema:

\begin{itemize}
	\item \textbf{Facilidade de Manutenção}: A manutenção das mensagens de erro torna-se mais simples, uma vez que as mensagens estão centralizadas num único local e não espalhadas pelo código.
	\item \textbf{Escalabilidade}: Novos erros podem ser facilmente adicionados à enumeração e ao dicionário de mensagens, sem a necessidade de reescrever as exceções ou a lógica de validação.
	\item \textbf{Internacionalização}: Como as mensagens de erro estão centralizadas, é mais fácil adaptar o sistema para diferentes idiomas no futuro.
	\item \textbf{Consistência}: Ao utilizar códigos de erro numéricos e associar cada um a uma mensagem específica, garantimos que o tratamento de exceções será consistente ao longo de toda a aplicação.
\end{itemize}

\newpage
\section{Tratamento de Logs}
\label{sec:tratamentologs}

O tratamento de logs é uma componente essencial para monitorizar o funcionamento do sistema, detetar erros e analisar o comportamento das operações realizadas. Nesta aplicação, utilizou-se a biblioteca \textit{Microsoft.Extensions.Logging}, que oferece uma solução integrada e extensível para registo de eventos. 

A abordagem adotada inclui a injeção direta do \textit{logger} através de dependência, garantindo que todas as classes que necessitam de registo de eventos podem utilizá-lo de forma centralizada e consistente.

\subsection{Injeção do Logger}

A injeção do logger é realizada diretamente no construtor da classe, utilizando a interface \texttt{ILogger<T>}, onde \texttt{T} é o tipo da classe onde o logger será utilizado. Este mecanismo permite associar o logger ao contexto específico da classe, proporcionando mensagens de log detalhadas e informativas.

O exemplo abaixo demonstra como o logger é injetado na classe \texttt{BookingManager}:

\vspace*{15pt}
\begin{lstlisting}[ 
	language=csh, 
	caption={Injeção do Logger no Construtor}, 
	label={lst:LoggerInjection}]
public BookingManager(ILogger<BookingManager> logger)
{
	_logger = logger ?? throw new ArgumentNullException(nameof(logger));
	_logger.LogInformation("BookingManager initialized.");
}
\end{lstlisting}

No exemplo acima, o logger é atribuído a uma variável privada e utilizado para registar uma mensagem de informação na inicialização da classe.

\subsection{Exemplo de Utilização}

O logger é usado para registar informações importantes durante a execução do sistema, tais como:

\begin{itemize}
	\item \textbf{Mensagens de informação} (\texttt{LogInformation}), para registar o estado e progresso de operações.
	\item \textbf{Mensagens de erro} (\texttt{LogError}), para reportar exceções e falhas.
	\item \textbf{Mensagens de \textit{debug}} (\texttt{LogDebug}), para análise detalhada durante o desenvolvimento.
\end{itemize}

Segue um exemplo do logger no código.

\vspace*{15pt}
\begin{lstlisting}[ 
	language=csh, 
	caption={Exemplo de Utilização do Logger}, 
	label={lst:LoggerUsage}]
public Client CreateBasicClient(string firstName, string lastName, string email)
{
	try
	{
		// Log information before creating the client
		_logger.LogInformation(
		"Attempting to create a new client with Name: {FirstName} {LastName}, Email: {Email}.", firstName,
		lastName, email);
		
		// Create a new client
		Client client = new Client(firstName, lastName, email); // May throw exception
		
		// Add client to the system
		_clients.Add(client);
		
		// Log success information
		_logger.LogInformation("Successfully created client: {FirstName} {LastName}, Email: {Email}.", firstName,
		lastName, email);
		
		// Return the created client
		return client;
	}
	catch (ValidationException ex)
	{
		// Log the exception with details
		_logger.LogError(ex, "Error while creating client with Name: {FirstName} {LastName}, Email: {Email}.",
		firstName, lastName, email);
		
		// Throw a new exception with more context
		throw new ClientCreationException("An error occurred while creating the client due to invalid input.", ex);
	}
}
\end{lstlisting}

Neste exemplo:

\begin{itemize}
	\item Antes de criar o cliente, é registada uma mensagem de informação com os detalhes fornecidos.
	\item Em caso de sucesso, uma mensagem de confirmação é registada.
	\item Se ocorrer uma exceção, é registada como erro, juntamente com os detalhes do contexto.
\end{itemize}

\subsection{Benefícios do Uso de Logs}

\begin{itemize}
	\item \textbf{Monitorização}: Os logs permitem monitorizar o estado da aplicação em tempo real, registando operações críticas e sucessos.
	\item \textbf{Diagnóstico de erros}: As mensagens de erro detalhadas ajudam na identificação e resolução de problemas rapidamente.
	\item \textbf{Auditoria}: O histórico de operações registado pelos logs permite analisar as ações realizadas.
	\item \textbf{Facilidade de depuração}: Durante o desenvolvimento, as mensagens de \textit{debug} podem fornecer informações detalhadas sobre o comportamento do sistema.
\end{itemize}

\subsection{Estratégias de Log no Sistema}

\begin{itemize}
	\item \textbf{Níveis de log}: Configurar os níveis de log (\texttt{Information}, \texttt{Debug}, \texttt{Error}) para evitar excesso de registos e manter a relevância das mensagens.
	\item \textbf{Persistência}: Utilizar \textit{providers} de log (ficheiros, bases de dados ou sistemas como o Serilog) para guardar logs persistentes, facilitando a análise a longo prazo.
	\item \textbf{Consistência}: Garantir que todas as classes seguem um padrão uniforme na utilização do logger, melhorando a legibilidade e a manutenção.
\end{itemize}

Com estas práticas, o sistema assegura um tratamento de logs robusto, contribuindo para a estabilidade, transparência e eficiência do desenvolvimento.

\newpage
\section{Persistência de Dados com Ficheiros}
\label{sec:persistenciadados}

A persistência de dados desempenha um papel crucial em qualquer sistema de gestão. Neste projeto, foi adotada a utilização de ficheiros JSON para a importação e exportação de dados, complementada por ficheiros binários baseados em \texttt{Protocol Buffers} (Protobuf). O formato JSON foi selecionado para a importação e exportação devido à sua popularidade e facilidade de utilização, especialmente em aplicações web, móveis e \textit{desktop}. Além disso, a sua legibilidade elevada torna-o ideal para a manipulação e \textit{debugging} de dados de forma simples e eficiente. Por outro lado, o formato binário com Protobuf foi escolhido para guardar e carregar dados, garantindo eficiência e desempenho superiores em termos de armazenamento e velocidade de processamento.

Para a persistência binária, foi escolhido o \texttt{Protobuf}, em vez dos métodos tradicionais binários como o \texttt{BinaryFormatter}, que é obsoleto e apresenta vulnerabilidades de segurança. O \texttt{BinaryFormatter} tem riscos significativos na desserialização de dados, o que pode resultar em falhas de segurança, como execução remota de código. O \texttt{Protobuf}, por outro lado, oferece uma solução mais segura, eficiente e compacta para armazenar dados de forma binária. Além disso, é compatível com diversas linguagens e plataformas, o que facilita a integração com outros sistemas.

Portanto, enquanto o JSON é utilizado para a leitura e manipulação de dados de forma mais acessível, o \texttt{Protobuf} é usado para a gravação e carregamento eficiente de dados binários, garantindo melhor desempenho e menor tamanho de ficheiro. Este equilíbrio entre formatos textuais e binários permite que o sistema aproveite o melhor de ambos os mundos: a simplicidade e flexibilidade do JSON para operações de importação/exportação e a eficiência do \texttt{Protobuf} para persistência binária.

\subsection{Importação e Exportação de Dados em JSON}

A importação e exportação de dados são funcionalidades essenciais para assegurar que a informação possa ser manipulada de forma eficaz em diferentes contextos. Neste projeto, optou-se pelo uso do formato JSON para estas operações, devido à sua legibilidade e ampla adoção em diversas plataformas. O formato JSON permite que os dados sejam facilmente lidos, editados e transferidos, sendo particularmente útil em ambientes que exigem interação com outras aplicações ou serviços.

\subsubsection{Exemplo de Importação de Dados em JSON}

O método \texttt{Import} permite importar uma lista de objetos a partir de uma string JSON. Caso a string JSON seja inválida ou nula, são lançadas exceções específicas para garantir que os dados sejam válidos antes de serem inseridos na coleção. O código seguinte apresenta um exemplo do método \textit{Import} dos clientes:

\begin{lstlisting}[
	language=csh, 
	caption={Método Import para Importar Clientes}, 
	label={lst:C}]
public ImportResult Import(string data)
{
	if (string.IsNullOrEmpty(data))
	{
		throw new ArgumentException("Data cannot be null or empty", nameof(data));
	}
	
	// Deserialize the data into a List<Client> instead of a single Client
	var clients = JsonHelper.DeserializeFromJson<Client>(data) ??
	throw new ArgumentException("Deserialized client data cannot be null", nameof(data));
	
	int replacedCount = 0;
	int importedCount = 0;
	
	foreach (var client in clients)
	{
		if (_clientDictionary.ContainsKey(client.Id))
		{
			replacedCount++;
		}
		else
		{
			importedCount++;
		}
		_clientDictionary[client.Id] = client; // Direct insertion for efficiency
	}
	
	return new ImportResult { ImportedCount = importedCount, ReplacedCount = replacedCount };
}
\end{lstlisting}

No método \texttt{Import}, são realizadas as seguintes validações:
\begin{itemize}
	\item Se a string \texttt{data} for nula ou vazia, uma exceção \texttt{ArgumentException} é lançada.
	\item A deserialização do JSON é realizada utilizando um método auxiliar (\texttt{DeserializeFromJson}), e, caso falhe, outra exceção \texttt{ArgumentException} é gerada.
	\item Para cada cliente na lista importada, o cliente é adicionado ao dicionário \texttt{\_clientDictionary}, substituindo qualquer cliente existente com o mesmo ID.
	\item O método \texttt{Import} retorna um objeto \texttt{ImportResult}, que contém a quantidade de clientes importados e substituídos, proporcionando informações úteis sobre o processo de importação.
\end{itemize}

\subsubsection{Exemplo de Exportação de Dados em JSON}

O método \texttt{Export} permite exportar a lista atual de clientes para uma string JSON, que pode ser armazenada num ficheiro ou enviada para outros sistemas. O código do método é o seguinte:

\begin{lstlisting}[
	language=csh, 
	caption={Método Export para Exportar Clientes},
	label={lst:C}]
public string Export()
{
	return JsonHelper.SerializeToJson(_clientDictionary.Values ?? Enumerable.Empty<Client>());
}
\end{lstlisting}

O método \texttt{Export} converte os valores guardados no dicionário \texttt{\_clientDictionary} numa string JSON utilizando o método auxiliar \texttt{SerializeToJson}. Se o dicionário estiver vazio, é retornada uma coleção vazia para garantir que o formato JSON seja válido.

Neste método, os valores do dicionário \texttt{\_clientDictionary} são serializados diretamente para uma string JSON. O uso de \texttt{Enumerable.Empty<Client>()} garante que, caso não existam clientes no dicionário, uma lista vazia seja serializada, evitando problemas com a estrutura JSON.

\subsubsection{Validação e Exceções na Persistência de Dados}

É importante garantir que os dados sejam consistentes e válidos antes de serem persistidos no formato JSON. Para isso, o sistema inclui verificações e exceções que asseguram que a importação e exportação dos dados sejam feitas corretamente.

As exceções tratadas nas operações de importação e exportação incluem:
\begin{itemize}
	\item \textbf{ArgumentException}: Lançada quando os dados fornecidos são nulos, vazios ou falham na desserialização.
	\item \textbf{InvalidOperationException}: Pode ser lançada em casos em que ocorre uma operação inesperada, como ao tentar aceder dados que não foram carregados corretamente.
\end{itemize}

Estas exceções permitem que o sistema lide com dados inválidos de forma controlada, sem causar falhas inesperadas. Além disso, a validação dos dados durante a importação e exportação impede que dados corrompidos ou malformados sejam persistidos, garantindo a integridade do sistema.

\subsection{Persistência de Dados em Formato Binário com Protobuf}

Embora o formato JSON seja simples e eficaz, em alguns casos, como sistemas com grandes volumes de dados ou requisitos de desempenho mais exigentes, pode ser vantajoso utilizar um formato binário mais compacto e eficiente. Nesse contexto, o \texttt{Protocol Buffers} (Protobuf) apresenta-se como uma alternativa poderosa.

Protobuf é um formato binário de serialização de dados desenvolvido pela Google, que permite uma codificação e decodificação mais rápida e compacta do que o JSON. Ao contrário do JSON, que é um formato de texto, o Protobuf armazena os dados de forma binária, o que reduz o tamanho do ficheiro e melhora o desempenho de operações de leitura e escrita, especialmente quando lidamos com muitos dados.

\subsubsection{Vantagens do Protobuf}

O uso do Protobuf para persistência de dados oferece várias vantagens:
\begin{itemize}
	\item \textbf{Eficiência de Espaço}: Como o Protobuf é um formato binário, ele é muito mais compacto em comparação ao JSON, o que resulta em menos espaço de armazenamento.
	\item \textbf{Desempenho Superior}: O Protobuf oferece tempos de leitura e escrita mais rápidos devido à sua natureza binária, tornando-o adequado para sistemas com alto tráfego de dados.
\end{itemize}

\subsubsection{Carregamento de Dados com Protobuf}

O método \texttt{Load} permite carregar uma coleção de clientes a partir de um ficheiro binário utilizando o formato \texttt{Protocol Buffers} (Protobuf). Caso ocorra algum erro durante o processo de leitura ou desserialização, são lançadas exceções específicas para diagnosticar o problema.

\begin{lstlisting}[
	language=csh, 
	caption={Método Load para Carregar Clientes com Protobuf}, 
	label={lst:protobuf-load}]
public void Load(string filePath)
{
	try
	{
		// Open the file stream for reading
		using (var fileStream = File.OpenRead(filePath))
		{
			// Deserialize the clients object from the file
			var clients = Serializer.Deserialize<Clients>(fileStream);
			
			// Copy the data from the deserialized object to the current instance
			_clientDictionary.Clear();
			foreach (var client in clients._clientDictionary)
			{
				_clientDictionary[client.Key] = client.Value;
			}
		}
	}
	catch (IOException ioEx)
	{
		throw new IOException("An error occurred while loading the clients data.", ioEx);
	}
	catch (SerializationException serEx)
	{
		throw new SerializationException("An error occurred during deserialization while loading the clients data.",
		serEx);
	}
}
\end{lstlisting}

O método \texttt{Load} realiza as seguintes operações:
\begin{itemize}
	\item Abre o ficheiro especificado para leitura, utilizando uma \textit{stream} do ficheiro.
	\item Desserializa os dados utilizando \texttt{Protocol Buffers}, recriando o objeto de clientes.
	\item Substitui os dados atuais na instância com os dados carregados do ficheiro.
	\item Lança exceções detalhadas em caso de falha na leitura ou na desserialização dos dados.
\end{itemize}

\subsubsection{Gravação de Dados com Protobuf}

O método \texttt{Save} permite guardar o estado atual da coleção de clientes num ficheiro binário utilizando o formato \texttt{Protocol Buffers} (Protobuf). Em caso de erro durante o processo de escrita ou serialização, são lançadas exceções adequadas.

\begin{lstlisting}[
	language=csh, 
	caption={Método Save para Guardar Clientes com Protobuf}, 
	label={lst:protobuf-save}]
public void Save(string filePath)
{
	try
	{
		// Prepare for serialization by copying the dictionary contents to the temporary list
		PrepareForSerialization();
		
		// Open the file stream for saving the data to the specified file
		using (var fileStream = File.Create(filePath))
		{
			// Serialize the clients object and write it to the file stream
			Serializer.Serialize(fileStream, this);
		}
	}
	catch (IOException ioEx)
	{
		throw new IOException("An error occurred while saving the clients data.", ioEx);
	}
	catch (SerializationException serEx)
	{
		throw new SerializationException("An error occurred during serialization while saving the clients data.",
		serEx);
	}
}
\end{lstlisting}

O método \texttt{Save} realiza as seguintes operações:
\begin{itemize}
	\item Prepara os dados para serialização, copiando os conteúdos do dicionário para uma estrutura temporária, caso necessário.
	\item Abre uma \textit{stream} do ficheiro no caminho especificado para guardar os dados.
	\item Serializa os dados no formato \texttt{Protocol Buffers} e escreve-os no ficheiro.
	\item Lança exceções detalhadas em caso de falha na escrita ou na serialização dos dados.
\end{itemize}

\subsubsection{Decoração de Classes para Protobuf}

Para utilizar o \texttt{Protocol Buffers} (Protobuf) como mecanismo de serialização, as classes a serem serializadas precisam de ser decoradas com atributos específicos fornecidos pela biblioteca. Estes atributos definem como os dados serão organizados e manipulados durante os processos de serialização e desserialização. No caso deste projeto, as classes como \texttt{Client} foram configuradas utilizando os atributos \texttt{[ProtoContract]} e \texttt{[ProtoMember]}.

\begin{lstlisting}[
	language=csh, 
	caption={Classe Client Decorada para Protobuf}, 
	label={lst:protobuf-decoration}]
[ProtoContract]
public class Client
{
	[ProtoMember(1)]
	readonly int _id; // ID of the client
	
	[ProtoMember(2)]
	string _firstName; // First name of the client
	
	[ProtoMember(3)]
	string _lastName; // Last name of the client
	
	(...)
\end{lstlisting}

\textbf{Atributos Principais}
\begin{itemize}
	\item \texttt{[ProtoContract]}: Este atributo marca a classe como serializável com Protobuf. Sem ele, a classe não será processada pelo mecanismo de Protobuf.
	\item \texttt{[ProtoMember(n)]}: Atributo utilizado para especificar a ordem dos campos durante a serialização. Cada membro deve ser associado a um número único que define sua posição na estrutura de dados serializada.
\end{itemize}

\newpage
\section{Testes Unitários}

\subsection{Introdução aos Testes Unitários}

Os testes unitários são uma parte essencial do desenvolvimento de software porque garantem que as funções e os métodos tenham o correto comporamento. O principal objetivo é validar o comportamento de cada parte do sistema de forma isolada, o que permite identificar e corrigir erros de maneira eficaz. Para implementar estes testes, usei o framework \texttt{xUnit}, que é uma escolha popular no ecossistema .NET devido à sua simplicidade e integração fácil com outras ferramentas.

\subsection{Razões para Utilizar Testes Unitários}

Os testes unitários ajudam a assegurar a qualidade do código, fornecendo uma camada extra de segurança ao desenvolvimento. No meu projeto, a utilização de testes permitiu:
\begin{itemize}
	\item \textbf{Detecção de erros}: Ao testar cada unidade de código de forma isolada, é possível identificar erros, evitando que se propaguem para outras partes do sistema.
	\item \textbf{Facilidade de refatoração}: Com testes automatizados, é possível refatorar o código com maior confiança, pois sabemos que os testes garantirão que as funcionalidades existentes não sejam comprometidas.
	\item \textbf{Documentação do comportamento esperado}: Os testes servem como uma forma de documentação do comportamento das funcionalidades, tornando o código mais compreensível.
\end{itemize}

\subsection{Implementação dos Testes com xUnit}

A escolha do \texttt{xUnit} foi motivada pela sua popularidade na comunidade .NET, bem como pela sua sintaxe simples e pela facilidade de integração com outros frameworks e ferramentas, como o \texttt{GitHub Actions} para CI/CD. Abaixo segue um exemplo de um teste unitário implementado com \texttt{xUnit}.

\begin{lstlisting}[
	language=csh, 
	caption={Teste do Construtor da Classe Client},
	label={lst:client-constructor-test}]
[Fact]
public void Constructor_ValidParameters_InitializesClient()
{
	// Arrange
	var firstName = "Enrique";
	var lastName = "Rodrigues";
	var email = "Enrique.Rodrigues@example.com";
	
	// Act
	var client = new Client(firstName, lastName, email);
	
	// Assert
	Assert.Equal(firstName, client.FirstName);
	Assert.Equal(lastName, client.LastName);
	Assert.Equal(email, client.Email);
	Assert.Equal(PaymentMethod.None, client.PreferredPaymentMethod); // Default value
	Assert.NotEqual(0, client.Id);                                   // ID should be auto-assigned
}
\end{lstlisting}

Este teste verifica o comportamento do construtor da classe \texttt{Client}, garantindo que os parâmetros fornecidos sejam corretamente atribuídos às propriedades da classe e que o valor do \texttt{ID} seja atribuído automaticamente.

\subsection{Integração com CI/CD}

A integração de testes unitários num pipeline CI/CD é uma prática recomendada para garantir que o código seja validado automaticamente em cada mudança ou atualização. No meu projeto, configurei um pipeline de integração contínua utilizando \texttt{GitHub Actions}, permitindo que os testes fossem executados automaticamente sempre que alterações fossem feitas no repositório.

O ficheiro \texttt{dotnet.yml}, que configura o pipeline de CI, é o seguinte:

\begin{lstlisting}[
	language=C, 
	caption={Configuração do Pipeline CI com GitHub Actions},
	label={lst:C}]
name: .NET CI

on:
push:
branches:
- main
pull_request:
branches:
- main

jobs:
build-and-test:
runs-on: ubuntu-latest

steps:
- name: Checkout code
uses: actions/checkout@v3

- name: Setup .NET
uses: actions/setup-dotnet@v3
with:
dotnet-version: 8.0.x

- name: Restore dependencies
working-directory: ./SmartStay
run: dotnet restore SmartStay.sln

- name: Build solution
working-directory: ./SmartStay
run: dotnet build SmartStay.sln --no-restore --configuration Release

- name: Run tests
working-directory: ./SmartStay
run: dotnet test --no-build --verbosity normal --configuration Release
\end{lstlisting}

Este ficheiro configura o pipeline para que, sempre que houver um \texttt{push} ou um \texttt{pull request} na branch \texttt{main}, o código seja compilado e os testes sejam executados de forma automática.

\subsection{Vantagens da Integração Contínua e Testes Automatizados}

Integrar os testes unitários com um pipeline CI/CD oferece várias vantagens:
\begin{itemize}
	\item \textbf{Automação}: A execução dos testes é automática sempre que há uma alteração no código, garantindo que nada seja esquecido.
	\item \textbf{Detecção rápida de falhas}: Caso algum teste falhe, o pipeline falha imediatamente, permitindo que os desenvolvedores corrijam os problemas de forma rápida.
	\item \textbf{Consistência}: Os testes são executados nas mesmas condições em cada execução do pipeline, assegurando que o comportamento do sistema seja consistente ao longo do tempo.
\end{itemize}

A utilização de testes unitários combinada com um pipeline de CI/CD robusto oferece uma solução eficiente e confiável para garantir a qualidade e a estabilidade do código durante todo o ciclo de desenvolvimento.

\newpage
\section{Desenvolvimento da Interface}
\label{sec:desenvolvimentoui}

\subsection{Introdução ao Razor Server App}

Para o desenvolvimento da interface do utilizador, optei por criar uma aplicação \textit{Razor Server App}. Esta escolha foi motivada pela minha experiência com \texttt{HTML} e \texttt{CSS}, tecnologias que já conhecia, e que me permitiram explorar as funcionalidades do Razor de forma eficiente. O \texttt{Razor} é uma engine de templates do .NET que facilita a criação de páginas dinâmicas ao integrar HTML com código C\# de maneira fluida. A sintaxe do Razor permite misturar de forma simples as estruturas de HTML com código C\#, oferecendo uma experiência de desenvolvimento mais rápida e intuitiva para aplicações web.

\subsection{Dependency Injection e Integração com o Backend}

No que diz respeito à arquitetura da aplicação, utilizei o \textit{Dependency Injection} (DI), uma prática recomendada no desenvolvimento de aplicações .NET para promover a inversão de controlo e facilitar a manutenção e testabilidade do código. O \textit{Dependency Injection} permite que os serviços e dependências da aplicação sejam injetados automaticamente nas classes que as necessitam, em vez de serem instanciados diretamente dentro dessas classes.

No meu projeto, o backend é composto por bibliotecas \texttt{DLL}, que são responsáveis pelas regras de negócio. Para integrar estas bibliotecas com o front-end Razor, configurei a injeção das dependências no \texttt{Program.cs}. O serviço \texttt{BookingManager}, responsável pela gestão das reservas, é injetado no container de serviços da aplicação da seguinte forma:

\begin{lstlisting}[
	language=csh, 
	caption={Configuração de Dependency Injection},
	label={lst:di-config}]
builder.Services.AddSingleton<BookingManager>();
\end{lstlisting}

Ao adicionar o \texttt{BookingManager} como um serviço singleton no container de DI, a aplicação garante que uma única instância da classe \texttt{BookingManager} seja criada e utilizada durante toda a execução da aplicação. Isto permite que o Razor tenha acesso às regras de negócio necessárias para processar as reservas, sem a necessidade de criar novas instâncias do \texttt{BookingManager} diretamente nas páginas Razor.

Esta abordagem facilita a gestão de dependências e contribui para uma maior organização e flexibilidade no código. Além disso, a injeção de dependências torna a aplicação mais fácil de testar, uma vez que podemos simular ou substituir as dependências nas classes de teste.

\subsection{Uso de Tailwind CSS e Flowbite}

Uma das escolhas importantes no desenvolvimento da interface do utilizador foi a substituição do \texttt{Bootstrap} pelo \texttt{Tailwind CSS}, uma framework de CSS altamente personalizável. A principal razão para esta escolha foi a flexibilidade que o \texttt{Tailwind} oferece em comparação com o \texttt{Bootstrap}, permitindo um controlo muito mais preciso sobre o design e estilo da aplicação. 

Enquanto o \texttt{Bootstrap} fornece um conjunto de componentes predefinidos com um estilo determinado, o \texttt{Tailwind} permite que se defina cada parte do layout utilizando uma abordagem de \textit{utility-first}. Com \texttt{Tailwind}, podemos aplicar classes utilitárias diretamente aos elementos HTML, o que resulta em código mais conciso e altamente configurável. Esta abordagem ajuda a evitar o uso excessivo de classes personalizadas e reduz a necessidade de escrever CSS adicional, tornando o processo de estilização mais ágil e eficiente.

\subsubsection{Integração com Flowbite}

Embora o \texttt{Tailwind CSS} forneça uma base sólida para a construção da interface, optei por integrar o \texttt{Flowbite}, uma biblioteca de componentes construídos em cima do \texttt{Tailwind}, para acelerar ainda mais o desenvolvimento. O \texttt{Flowbite} oferece uma série de componentes prontos para uso, como botões, modais, tabelas, menus de navegação e muito mais, todos desenhados para se integrarem perfeitamente com o \texttt{Tailwind}. 

O uso do \texttt{Flowbite} não só poupou tempo no desenvolvimento, mas também garantiu que a interface tivesse um design moderno e funcional, com componentes que seguem as melhores práticas de design responsivo e acessibilidade. Além disso, como o \texttt{Flowbite} é construído em cima do \texttt{Tailwind}, não houve conflito entre os estilos e foi fácil de personalizar os componentes conforme as necessidades do projeto.

Para integrar o \texttt{Flowbite} com o projeto, foi necessário instalar a biblioteca e configurá-la no ficheiro de recursos do projeto. O processo de instalação foi simples e envolveu a adição do \texttt{Flowbite} através de \texttt{npm}. Abaixo está um exemplo de como a integração foi realizada no ficheiro \_Layout.cshtml:

\begin{lstlisting}[
	language=html, 
	caption={Integração do Flowbite},
	label={lst:flowbite-integration}]

<!DOCTYPE html>
<html lang="en">
<head>
	<meta charset="utf-8" />
	<meta name="viewport" content="width=device-width, initial-scale=1.0" />
	<base href="~/" />
	<link href="~/css/app.min.css" rel="stylesheet" />
	<link href="SmartStay.App.styles.css" rel="stylesheet" />
	<component type="typeof(HeadOutlet)" render-mode="ServerPrerendered" />
</head>
<body>
	@RenderBody()
	
	<div id="blazor-error-ui">
		<environment include="Staging,Production">
		An error has occurred. This application may no longer respond until reloaded.
		</environment>
		<environment include="Development">
		An unhandled exception has occurred. See browser dev tools for details.
		</environment>
		<a href="" class="reload">Reload</a>
		<a class="dismiss">X</a>
	</div>
	
	<script src="_framework/blazor.server.js"></script>
	<script 
	src="https://cdn.jsdelivr.net/npm/flowbite@2.5.2/dist/flowbite.min.js">
	</script>
</body>
</html>
\end{lstlisting}

Com esta integração, foi possível melhorar a experiência do utilizador e otimizar o tempo de desenvolvimento, aproveitando os componentes de UI prontos do \texttt{Flowbite} enquanto ainda mantinha a flexibilidade do \texttt{Tailwind}.

\newpage
\section{Conclusão}
\label{sec:conclusion}

Lorem Ipsum.

%%% ===============================================================================
%%% Bibliography
%%% ===============================================================================

\renewcommand{\bibsection}{\section*{Referências}} % requried for natbib to have "References" printed and as section*, not chapter*
% Use natbib compatbile splncs04nat style.
% It does provide all features of splncs04.bst, but is developed in a clean way.
% Source: https://github.com/tpavlic/splncs04nat
\bibliographystyle{splncs04nat}
\begingroup
  \microtypecontext{expansion=sloppy}
  \small % ensure correct font size for the bibliography
  \bibliography{paper}
\endgroup

% Enfore empty line after bibliography
\ \\
%
Todas as ligações foram consultadas pela última vez no dia 15 de novembro de 2024.


%%% ===============================================================================
%\appendix
%\addcontentsline{toc}{chapter}{APPENDICES}

%\listoffigures
%\listoftables
%%% ===============================================================================

%%% ===============================================================================
%\section{My first appendix}\label{sec:appendix1}
%%% ===============================================================================
\end{document}
