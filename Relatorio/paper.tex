% This template has been tested with LLNCS DOCUMENT CLASS -- version 2.21 (12-Jan-2022)

% !TeX spellcheck = pt_PT-Portuguese
% !LTeX: language= pt_PT-Portuguese
% !TeX encoding = utf8
% !TeX program = pdflatex
% !BIB program = bibtex
% -*- coding:utf-8 mod:LaTeX -*-

% "a4paper" enables:
%
%  - easy print out on DIN A4 paper size
%
% One can configure a4 vs. letter in the LaTeX installation. So it is configuration dependend, what the paper size will be.
% This option  present, because the current word template offered by Springer is DIN A4.
% We accept that DIN A4 cause WTFs at persons not used to A4 in USA.
%
% "runningheads" enables:
%
%  - page number on page 2 onwards
%  - title/authors on even/odd pages
%
% This is good for other readers to enable proper archiving among other papers and pointing to
% content. Even if the title page states the title, when printed and stored in a folder, when
% blindly opening the folder, one could hit not the title page, but an arbitrary page. Therefore,
% it is good to have title printed on the pages, too.
%
% To disable outputting page headers and footers, remove "runningheads"
\documentclass[runningheads,a4paper,english]{llncs}[2022/01/12]

% backticks (`) are rendered as such in verbatim environments.
% See following links for details:
%   - https://tex.stackexchange.com/a/341057/9075
%   - https://tex.stackexchange.com/a/47451/9075
%   - https://tex.stackexchange.com/a/166791/9075
\usepackage{upquote}

% Used to add images
\usepackage{graphicx}

% Used for figures
\usepackage{float}

% Set English as language and allow to write hyphenated"=words
%
% Even though `american`, `english` and `USenglish` are synonyms for babel package (according to https://tex.stackexchange.com/questions/12775/babel-english-american-usenglish), the llncs document class is prepared to avoid the overriding of certain names (such as "Abstract." -> "Abstract" or "Fig." -> "Figure") when using `english`, but not when using the other 2.
% english has to go last to set it as default language
\usepackage[ngerman,main=english]{babel}
%
% Hint by http://tex.stackexchange.com/a/321066/9075 -> enable "= as dashes
\addto\extrasenglish{\languageshorthands{ngerman}\useshorthands{"}}
%
% Fix by https://tex.stackexchange.com/a/441701/9075
\usepackage{regexpatch}
\makeatletter
\edef\switcht@albion{%
  \relax\unexpanded\expandafter{\switcht@albion}%
}
\xpatchcmd*{\switcht@albion}{ \def}{\def}{}{}
\xpatchcmd{\switcht@albion}{\relax}{}{}{}
\edef\switcht@deutsch{%
  \relax\unexpanded\expandafter{\switcht@deutsch}%
}
\xpatchcmd*{\switcht@deutsch}{ \def}{\def}{}{}
\xpatchcmd{\switcht@deutsch}{\relax}{}{}{}
\edef\switcht@francais{%
  \relax\unexpanded\expandafter{\switcht@francais}%
}
\xpatchcmd*{\switcht@francais}{ \def}{\def}{}{}
\xpatchcmd{\switcht@francais}{\relax}{}{}{}
\makeatother

% Links behave as they should. Enables "\url{...}" for URL typesettings.
% Allow URL breaks also at a hyphen, even though it might be confusing: Is the "-" part of the address or just a hyphen?
% See https://tex.stackexchange.com/a/3034/9075.
\usepackage[hyphens]{url}

% When activated, use text font as url font, not the monospaced one.
% For all options see https://tex.stackexchange.com/a/261435/9075.
% \urlstyle{same}

% Improve wrapping of URLs - hint by http://tex.stackexchange.com/a/10419/9075
\makeatletter
\g@addto@macro{\UrlBreaks}{\UrlOrds}
\makeatother

% nicer // - solution by http://tex.stackexchange.com/a/98470/9075
% DO NOT ACTIVATE -> prevents line breaks
%\makeatletter
%\def\Url@twoslashes{\mathchar`\/\@ifnextchar/{\kern-.2em}{}}
%\g@addto@macro\UrlSpecials{\do\/{\Url@twoslashes}}
%\makeatother

% This is the modern package for "Computer Modern".
% In case this gets activated, one has to switch from cmap package to glyphtounicode (in the case of pdflatex)
\usepackage[%
    rm={oldstyle=false,proportional=true},%
    sf={oldstyle=false,proportional=true},%
    % By using 'variable=true' the monospaced font can be used as variable font (with differents widths per letter)
    % However, this makes listings look ugly.
    tt={oldstyle=false,proportional=true,variable=false},%
    qt=false%
]{cfr-lm}

% Has to be loaded AFTER any font packages. See https://tex.stackexchange.com/a/2869/9075.
\usepackage[T1]{fontenc}

% Character protrusion and font expansion. See http://www.ctan.org/tex-archive/macros/latex/contrib/microtype/

\usepackage[
  babel=true, % Enable language-specific kerning. Take language-settings from the languge of the current document (see Section 6 of microtype.pdf)
  expansion=alltext,
  protrusion=alltext-nott, % Ensure that at listings, there is no change at the margin of the listing
  final % Always enable microtype, even if in draft mode. This helps finding bad boxes quickly.
        % In the standard configuration, this template is always in the final mode, so this option only makes a difference if "pros" use the draft mode
]{microtype}

% \texttt{test -- test} keeps the "--" as "--" (and does not convert it to an en dash)
\DisableLigatures{encoding = T1, family = tt* }

%\DeclareMicrotypeSet*[tracking]{my}{ font = */*/*/sc/* }%
%\SetTracking{ encoding = *, shape = sc }{ 45 }
% Source: http://homepage.ruhr-uni-bochum.de/Georg.Verweyen/pakete.html
% Deactiviated, because does not look good

\usepackage{graphicx}

% Diagonal lines in a table - http://tex.stackexchange.com/questions/17745/diagonal-lines-in-table-cell
% Slashbox is not available in texlive (due to licensing) and also gives bad results. Thus, we use diagbox
\usepackage{diagbox}

\usepackage{xcolor}

% Code Listings
\usepackage{listings}

\definecolor{eclipseStrings}{RGB}{42,0.0,255}
\definecolor{eclipseKeywords}{RGB}{127,0,85}
\colorlet{numb}{magenta!60!black}

% JSON definition
% Source: https://tex.stackexchange.com/a/433961/9075

\lstdefinelanguage{json}{
    basicstyle=\normalfont\ttfamily,
    commentstyle=\color{eclipseStrings}, % style of comment
    stringstyle=\color{eclipseKeywords}, % style of strings
    numbers=left,
    numberstyle=\scriptsize,
    stepnumber=1,
    numbersep=8pt,
    showstringspaces=false,
    breaklines=true,
    frame=lines,
    % backgroundcolor=\color{gray}, %only if you like
    string=[s]{"}{"},
    comment=[l]{:\ "},
    morecomment=[l]{:"},
    literate=
        *{0}{{{\color{numb}0}}}{1}
         {1}{{{\color{numb}1}}}{1}
         {2}{{{\color{numb}2}}}{1}
         {3}{{{\color{numb}3}}}{1}
         {4}{{{\color{numb}4}}}{1}
         {5}{{{\color{numb}5}}}{1}
         {6}{{{\color{numb}6}}}{1}
         {7}{{{\color{numb}7}}}{1}
         {8}{{{\color{numb}8}}}{1}
         {9}{{{\color{numb}9}}}{1}
}

\lstset{
  % everything between (* *) is a latex command
  escapeinside={(*}{*)},
  %
  language=json,
  %
  showstringspaces=false,
  %
  extendedchars=true,
  %
  basicstyle=\footnotesize\ttfamily,
  %
  commentstyle=\slshape,
  %
  % default: \rmfamily
  stringstyle=\ttfamily,
  %
  breaklines=true,
  %
  breakatwhitespace=true,
  %
  % alternative: fixed
  columns=flexible,
  %
  numbers=left,
  %
  numberstyle=\tiny,
  %
  basewidth=.5em,
  %
  xleftmargin=.5cm,
  %
  % aboveskip=0mm,
  %
  % belowskip=0mm,
  %
  captionpos=b
}

% Enable Umlauts when using \lstinputputlisting.
% See https://stackoverflow.com/a/29260603/873282 für details.
% listingsutf8 did not work in June 2020.
\lstset{literate=
  {á}{{\'a}}1 {é}{{\'e}}1 {í}{{\'i}}1 {ó}{{\'o}}1 {ú}{{\'u}}1
  {Á}{{\'A}}1 {É}{{\'E}}1 {Í}{{\'I}}1 {Ó}{{\'O}}1 {Ú}{{\'U}}1
  {à}{{\`a}}1 {è}{{\`e}}1 {ì}{{\`i}}1 {ò}{{\`o}}1 {ù}{{\`u}}1
  {À}{{\`A}}1 {È}{{\'E}}1 {Ì}{{\`I}}1 {Ò}{{\`O}}1 {Ù}{{\`U}}1
  {ä}{{\"a}}1 {ë}{{\"e}}1 {ï}{{\"i}}1 {ö}{{\"o}}1 {ü}{{\"u}}1
  {Ä}{{\"A}}1 {Ë}{{\"E}}1 {Ï}{{\"I}}1 {Ö}{{\"O}}1 {Ü}{{\"U}}1
  {â}{{\^a}}1 {ê}{{\^e}}1 {î}{{\^i}}1 {ô}{{\^o}}1 {û}{{\^u}}1
  {Â}{{\^A}}1 {Ê}{{\^E}}1 {Î}{{\^I}}1 {Ô}{{\^O}}1 {Û}{{\^U}}1
  {Ã}{{\~A}}1 {ã}{{\~a}}1 {Õ}{{\~O}}1 {õ}{{\~o}}1
  {œ}{{\oe}}1 {Œ}{{\OE}}1 {æ}{{\ae}}1 {Æ}{{\AE}}1 {ß}{{\ss}}1
  {ű}{{\H{u}}}1 {Ű}{{\H{U}}}1 {ő}{{\H{o}}}1 {Ő}{{\H{O}}}1
  {ç}{{\c c}}1 {Ç}{{\c C}}1 {ø}{{\o}}1 {å}{{\r a}}1 {Å}{{\r A}}1
}

% For easy quotations: \enquote{text}
% This package is very smart when nesting is applied, otherwise textcmds (see below) provides a shorter command
\usepackage[autostyle=true]{csquotes}

% Enable using "`quote"' - see https://tex.stackexchange.com/a/150954/9075
\defineshorthand{"`}{\openautoquote}
\defineshorthand{"'}{\closeautoquote}

% Nicer tables (\toprule, \midrule, \bottomrule)
\usepackage{booktabs}

% Extended enumerate, such as \begin{compactenum}
\usepackage{paralist}

% Bibliopgraphy enhancements
%  - enable \cite[prenote][]{ref}
%  - enable \cite{ref1,ref2}
% Alternative: \usepackage{cite}, which enables \cite{ref1, ref2} only (otherwise: Error message: "White space in argument")

% Doc: http://texdoc.net/natbib
\usepackage[%
  square,        % for square brackets
  comma,         % use commas as separators
  numbers,       % for numerical citations;
  %sort           % orders multiple citations into the sequence in which they appear in the list of references;
  sort&compress  % as sort but in addition multiple numerical citations
                 % are compressed if possible (as 3-6, 15);
]{natbib}

% In the bibliography, references have to be formatted as 1., 2., ... not [1], [2], ...
\renewcommand{\bibnumfmt}[1]{#1.}

% Enable hyperlinked author names in the case of \citet
% Source: https://tex.stackexchange.com/a/76075/9075
\usepackage{etoolbox}
\makeatletter
\patchcmd{\NAT@test}{\else \NAT@nm}{\else \NAT@hyper@{\NAT@nm}}{}{}
\makeatother

% Prepare more space-saving rendering of the bibliography
% Source: https://tex.stackexchange.com/a/280936/9075
\SetExpansion
[ context = sloppy,
  stretch = 30,
  shrink = 60,
  step = 5 ]
{ encoding = {OT1,T1,TS1} }
{ }

% Put figures aside a text
\usepackage[rflt]{floatflt}

% Enable nice comments
\usepackage{pdfcomment}

\newcommand{\commentontext}[2]{\colorbox{yellow!60}{#1}\pdfcomment[color={0.234 0.867 0.211},hoffset=-6pt,voffset=10pt,opacity=0.5]{#2}}
\newcommand{\commentatside}[1]{\pdfcomment[color={0.045 0.278 0.643},icon=Note]{#1}}

% Compatibality with packages todo, easy-todo, todonotes
\newcommand{\todo}[1]{\commentatside{#1}}

% Compatiblity with package fixmetodonotes
\newcommand{\TODO}[1]{\commentatside{#1}}

% Put footnotes below floats
% Source: https://tex.stackexchange.com/a/32993/9075
\usepackage{stfloats}
\fnbelowfloat

\usepackage[group-minimum-digits=4,per-mode=fraction]{siunitx}

% Enable that parameters of \cref{}, \ref{}, \cite{}, ... are linked so that a reader can click on the number an jump to the target in the document
\usepackage{hyperref}

% Enable hyperref without colors and without bookmarks
\hypersetup{
  hidelinks,
  colorlinks=true,
  allcolors=black,
  pdfstartview=Fit,
  breaklinks=true
}

% Enable correct jumping to figures when referencing
\usepackage[all]{hypcap}

\usepackage[caption=false,font=footnotesize]{subfig}

\usepackage{mindflow}

% Extensions for references inside the document (\cref{fig:sample}, ...)
% Enable usage \cref{...} and \Cref{...} instead of \ref: Type of reference included in the link
% That means, "Figure 5" is a full link instead of just "5".
\usepackage[capitalise,nameinlink]{cleveref}

\crefname{section}{Sec.}{Sec.}
\Crefname{section}{Section}{Sections}
\crefname{listing}{List.}{List.}
\crefname{listing}{Listing}{Listings}
\Crefname{listing}{Listing}{Listings}
\crefname{lstlisting}{Listing}{Listings}
\Crefname{lstlisting}{Listing}{Listings}

\usepackage{lipsum}

% For demonstration purposes only
% These packages can be removed when all examples have been deleted
\usepackage[math]{blindtext}
\usepackage{mwe}
\usepackage[realmainfile]{currfile}
\usepackage{tcolorbox}
\tcbuselibrary{listings}

%introduce \powerset - hint by http://matheplanet.com/matheplanet/nuke/html/viewtopic.php?topic=136492&post_id=997377
\DeclareFontFamily{U}{MnSymbolC}{}
\DeclareSymbolFont{MnSyC}{U}{MnSymbolC}{m}{n}
\DeclareFontShape{U}{MnSymbolC}{m}{n}{
  <-6>    MnSymbolC5
  <6-7>   MnSymbolC6
  <7-8>   MnSymbolC7
  <8-9>   MnSymbolC8
  <9-10>  MnSymbolC9
  <10-12> MnSymbolC10
  <12->   MnSymbolC12%
}{}
\DeclareMathSymbol{\powerset}{\mathord}{MnSyC}{180}

\usepackage{xspace}
\newcommand{\eg}{e.g.,\ }
\newcommand{\ie}{i.e.,\ }

% Enable hyphenation at other places as the dash.
% Example: applicaiton\hydash specific
\makeatletter
\newcommand{\hydash}{\penalty\@M-\hskip\z@skip}
% Definition of "= taken from http://mirror.ctan.org/macros/latex/contrib/babel-contrib/german/ngermanb.dtx
\makeatother

% Add manual adapted hyphenation of English words
% See https://ctan.org/pkg/hyphenex and https://tex.stackexchange.com/a/22892/9075 for details
% Does not work on MiKTeX, therefore disabled - issue reported at https://github.com/MiKTeX/miktex-packaging/issues/271
% \input{ushyphex}

% correct bad hyphenation here
\hyphenation{op-tical net-works semi-conduc-tor}

% Add copyright
%
% This is recommended if you intend to send the version to colleagues
% See https://ctan.org/pkg/llncsconf for details
\iffalse
  % state: intended | submitted | llncs
  % you can add "crop" if the paper should be cropped to the format Springer is publishing
  \usepackage[intended]{llncsconf}

  \conference{name of the conference}

  % in case of "proceedings" (final version!)
  % example: \llncs{Anonymous et al. (eds). \emph{Proceedings of the International Conference on \LaTeX-Hacks}, LNCS~42. Some Publisher, 2016.}{0042}
  % 0042 denotes an example start page
  \llncs{book editors and title}{0042}
\fi

% Enable copy and paste of text from the PDF
% Only required for pdflatex. It "just works" in the case of lualatex.
% Alternative: cmap or mmap package
% mmap enables mathematical symbols, but does not work with the newtx font set
% See: https://tex.stackexchange.com/a/64457/9075
% Other solutions outlined at http://goemonx.blogspot.de/2012/01/pdflatex-ligaturen-und-copynpaste.html and http://tex.stackexchange.com/questions/4397/make-ligatures-in-linux-libertine-copyable-and-searchable
% Trouble shooting outlined at https://tex.stackexchange.com/a/100618/9075
%
% According to https://tex.stackexchange.com/q/451235/9075 this is the way to go
\input glyphtounicode
\pdfgentounicode=1

\begin{document}

\title{Gestão de Alojamentos Turísticos}
% If Title is too long, use \titlerunning
%\titlerunning{Short Title}
\subtitle{Programação Orientada a Objetos}

% Single insitute
\author{Enrique George Rodrigues Nº 28602}

% If there are too many authors, use \authorrunning
%\authorrunning{First Author et al.}

\institute{Instituto Politécnico do Cávado e do Ave\\
	Licenciatura Engenharia de Sistemas Informáticos\\
	\vspace{6pt}
	16 de Outubro de 2024}

%% Multiple insitutes - ALTERNATIVE to the above
% \author{%
%     Firstname Lastname\inst{1} \and
%     Firstname Lastname\inst{2}
% }
%
%If there are too many authors, use \authorrunning
%  \authorrunning{First Author et al.}
%
%  \institute{
%      Insitute 1\\
%      \email{...}\and
%      Insitute 2\\
%      \email{...}
%}

\maketitle

\renewcommand{\abstractname}{Resumo.} % Change the abstract title to "Resumo"
\begin{abstract}
	
Este projeto tem como objetivo o desenvolvimento de uma aplicação para a gestão de alojamentos turísticos, permitindo a administração eficiente de registos, reservas e consultas relacionadas com clientes e alojamentos. A solução foi implementada em C\#, utilizando o paradigma de programação orientada a objetos, de forma a garantir uma estrutura modular e escalável. As principais funcionalidades incluem o registo de clientes, a gestão de reservas, o processamento de check-ins e check-outs, bem como o controlo de pagamentos. A aplicação destaca-se pelo uso de dicionários como estruturas de dados eficientes, permitindo um acesso rápido e otimizado a registos de clientes, reservas e alojamentos, com uma complexidade temporal de $O(1)$ para operações de consulta e inserção. A persistência de dados é gerida em formato JSON, proporcionando uma solução leve e prática para operadores turísticos.

\renewcommand{\abstractname}{Resumo}

% Redefine keywords manually since llncs doesn't define \keywordsname
\def\keywords{\vspace{.5em}
	{\textbf{Palavras-chave:}}\,\relax%
}

\keywords{registos, consultas, reservas, check-in, clientes, alojamentos}
\end{abstract}

\newpage
\setcounter{page}{1}
\section{Introdução}
\label{sec:introduction}

O setor do turismo enfrenta desafios crescentes na gestão eficiente de alojamentos, exigindo soluções que garantam agilidade, precisão e uma experiência satisfatória para os clientes. Este projeto visa desenvolver um sistema de gestão de alojamentos turísticos que permita a administração eficaz de registos de clientes, reservas e alojamentos. A implementação foi realizada em C\#, uma linguagem reconhecida pela sua robustez e flexibilidade, permitindo o desenvolvimento de aplicações escaláveis e eficientes.

Para garantir um desempenho otimizado, o sistema utiliza dicionários como estruturas de dados principais, que se baseiam em tabelas de hash, proporcionando acesso rápido e eficiente a informações cruciais, com uma complexidade temporal de $O(1)$ para operações de consulta e inserção. Além disso, a persistência de dados é gerida em formato JSON, facilitando a integração e manipulação das informações. Embora a eficiência e a funcionalidade sejam os pilares deste projeto, a aplicação também foi concebida com atenção às boas práticas de programação, promovendo um código limpo e manutenível. Desta forma, o sistema não apenas atende às necessidades atuais do setor, mas também está preparado para futuras evoluções e inovações na gestão de alojamentos turísticos.

Este documento descreve o trabalho prático da unidade curricular de Programação Orientada a Objetos, parte integrante da Licenciatura em Engenharia de Sistemas Informáticos no Instituto Politécnico do Cávado e do Ave. 

O enunciado deste trabalho prático pode ser encontrado no anexo titulado ``Trabalho\_POO\_ESI\_2024\_2025``. O código fonte está disponível no Github, veja \href[]{LINK}.

\section{Padrões e Práticas de Programação}
\label{sec:codingstyle}

Para garantir a consistência, legibilidade e manutenção da base de código, este projeto segue os padrões de programação estabelecidos. Estes padrões estão alinhados com o Guia de Estilo da Microsoft para C\# \cite{microsoftcsharpstyle}, promovendo uniformidade em todo o projeto.

\subsection{Convenções de Nomeação}

A adoção de uma convenção de nomeação clara melhora a clareza do código. As seguintes convenções foram utilizadas ao longo do projeto:

\begin{itemize}
	\item \textbf{Classes e Structs}: \texttt{ClassName}
	\item \textbf{Métodos}: \texttt{MethodName()}
	\item \textbf{Propriedades}: \texttt{PropertyName}
	\item \textbf{Variáveis de Instância e Locais}: \texttt{variableName}
	\item \textbf{Constantes}: \texttt{CONSTANT\_NAME}
	\item \textbf{Parâmetros}: \texttt{parameterName}
	\item \textbf{Enums}: \texttt{EnumName}
\end{itemize}

Evita-se o uso de abreviações excessivas e nomes confusos, promovendo a clareza no código.

\subsection{Regras de Formatação}

A formatação consistente é fundamental para a legibilidade:

\begin{itemize}
	\item \textbf{Indentação}: 4 espaços por nível de indentação, evitando tabs.
	\item \textbf{Comprimento da Linha}: Linhas limitadas a um máximo de 120 caracteres.
	\item \textbf{Chaves}: As chaves de abertura são colocadas na próxima linha da declaração de controle.
	\item \textbf{Espaçamento}: Espaços em branco são usados para separar operadores e ao redor de chaves.
	\item \textbf{Comentários}: Comentários descritivos são utilizados para explicar seções complexas do código, seguindo um estilo consistente.
\end{itemize}

\subsection{Práticas de Programação}

Seguir as melhores práticas garante a qualidade do código:

\begin{itemize}
	\item Utilizam-se nomes claros para variáveis e métodos, tornando o código autoexplicativo.
	\item Escreve-se código claro e conciso, evitando complexidade desnecessária.
	\item Segue-se o Guia de Estilo da Microsoft para C\# \cite{microsoftcsharpstyle} para formatação e estilo gerais, garantindo a conformidade com os padrões da indústria.
\end{itemize}

\subsection{Controlo de Versões}

Utiliza-se Git para o controle de versões, facilitando a gestão eficaz da base de código:

\begin{itemize}
	\item Cria-se uma nova branch para cada funcionalidade ou correção de erro.
	\item As branches são criadas a partir da branch principal para desenvolvimento, com nomes descritivos que refletem a funcionalidade ou erro abordado.
	\item Evita-se fazer commits diretamente na branch principal, garantindo a estabilidade.
	\item As branches de funcionalidade são regularmente fundidas de volta na branch principal após testes completos.
\end{itemize}

\subsection{Documentação}

Utiliza-se \textit{XML Documentation Comments} para documentar o código, permitindo a geração automática de documentação da API:

\begin{itemize}
	\item Segue-se a sintaxe do XML para comentários, documentando métodos, propriedades e classes.
	\item A documentação inclui descrições breves, descrições de parâmetros, descrições de valores de retorno e exemplos de uso.
	\item Utilizam-se tags como \texttt{<summary>}, \texttt{<param>}, \texttt{<returns>} e \texttt{<example>} para estruturar corretamente os comentários.
	\item Define-se o autor nos comentários de cabeçalho de classes para novos ficheiros ou seções significativas de código.
\end{itemize}

\section{Estrutura de Classes}
\label{sec:classes}

\subsection{Introdução ao Sistema}
O diagrama de classes apresentado descreve a estrutura lógica do sistema de gestão de reservas para alojamentos turísticos, com uma arquitetura desenhada para centralizar e organizar informações essenciais sobre clientes, reservas, alojamentos e pagamentos. Este modelo, por além de otimizar o fluxo de trabalho, promove a extensibilidade e modularidade, facilitando tanto as expansões futuras como a manutenção do sistema.

\subsection{Classes Principais}
No diagrama, cada classe representa uma componente fundamental do sistema. A classe \textit{Client} (cliente) armazena informações sobre um cliente individual, como nome, e-mail e endereço. Para gerir vários clientes, o sistema utiliza a classe \textit{Clients}, que funciona como uma coleção para armazenar e organizar várias instâncias da classe \textit{Client}. Esta coleção é implementada através de um dicionário, permitindo o acesso eficiente e rápido aos dados dos clientes.

A classe \textit{Accommodation} (alojamento) representa as propriedades ou quartos disponíveis para reserva, com informações como o tipo de alojamento, a capacidade e o preço por noite. Parecida à classe \textit{Clients}, o sistema utiliza a classe \textit{Accommodations}, que também serve como uma coleção para armazenar e gerir vários alojamentos, sendo implementada através de um dicionário para facilitar a consulta e a manipulação dos dados.

A classe \textit{Reservation} (reserva) liga os clientes aos alojamentos, registando os detalhes da estadia, incluindo as datas de check-in e check-out, o custo total e o estado do pagamento. Tal como as classes \textit{Clients} e \textit{Accommodations}, o sistema utiliza a classe \textit{Reservations}, que guarda uma coleção de reservas, também gerida por um dicionário, permitindo a manipulação eficiente de múltiplas instâncias de reservas.

\subsection{Processamento de Pagamentos}
A classe \textit{Payment} (pagamento) tem como objetivo guardar informações detalhadas sobre cada transação associada a uma reserva, incluindo o montante pago, a data do pagamento e o método utilizado. A classe \textit{Reservation} mantém uma coleção de instâncias da classe \textit{Payment}, permitindo registar todos os pagamentos realizados para uma determinada reserva.

\subsection{BookingManager e a Interface ManageableEntity}
A classe \textit{BookingManager} (gestor de reservas) atua como o controlador principal do sistema, fornecendo métodos para adicionar e remover clientes, criar e cancelar reservas, gerir alojamentos e processar pagamentos.

O sistema introduz a interface \textit{ManageableEntity}, implementada pelas classes \textit{Clients}, \textit{Reservations} e \textit{Accommodations}. Esta interface unifica os métodos básicos de gestão, como \textit{Add()}, \textit{Remove()}, \textit{Import()} e \textit{Export()}, permitindo que o \textit{BookingManager} interaja com estas classes de forma uniforme. 

\subsection{Padrão Facade}
O \textit{BookingManager} atua como a face do sistema, permitindo direcionar chamadas para a classe apropriada com base no tipo de entidade a ser manipulada. Isto segue o \textit{Padrão Facade} \cite{facadepattern}, que simplifica a interação com o sistema ao fornecer uma interface única e centralizada. Este padrão oculta as complexidades internas do sistema, como a gestão dos dados em várias coleções, e permite que o utilizador ou outros componentes do sistema façam operações de forma mais intuitiva e eficiente. Ao centralizar a lógica de interação, o \textit{Facade} promove uma maior modularidade e extensibilidade, uma vez que mudanças internas podem ser feitas sem impactar a interface externa. Desta forma, o \textit{BookingManager} oferece um ponto único de acesso para a criação, manipulação e gestão das entidades, facilitando a manutenção e escalabilidade do sistema.

\begin{figure}[h!]
	\centering
	\includegraphics[width=1.0\textwidth]{Images/POO_Class_Diagram.png}
	\caption{Diagrama de Classes do Sistema de Gestão de Alojamentos Turísticos}
	\label{fig:diagrama_classes}
\end{figure}

\section{Estruturas de Dados Utilizadas}
\label{sec:estruturasdados}

\subsection{Utilização de Dictionary}
As estruturas de dados escolhidas para o sistema foram definidas com o objetivo de otimizar a eficiência e o desempenho, especialmente em operações de busca, inserção e remoção. A principal estrutura de dados utilizada foi o \textit{Dictionary} (dicionário), que oferece uma maneira eficiente de armazenar e aceder a entidades como clientes, reservas e alojamentos, por meio de chaves únicas. Esta estrutura garante um tempo de acesso quase constante para operações como inserção e busca, o que é essencial para o bom desempenho do sistema, especialmente à medida que o número de entidades cresce.

\vspace*{15pt}
\renewcommand{\lstlistingname}{Amostra de Código}
\begin{lstlisting}[ 
	language=csh, 
	caption={Uso de Dictionary}, 
	label={lst:C}]
readonly Dictionary<int, Client> _clientDictionary = new Dictionary<int, Client>();
\end{lstlisting}

O uso do \textit{Dictionary} garante que as operações de inserção e remoção de clientes sejam realizadas de forma rápida, com um tempo médio de execução constante $O(1)$, o que melhora a escalabilidade do sistema à medida que mais clientes são adicionados.

\subsection{Utilização de List e Pesquisa Binária}

Para a classe \textit{Reservation}, foi utilizada uma \textit{List} de tuplas para armazenar intervalos de datas de reservas, o que permite gerir as datas de forma eficiente. A lista é ordenada, e para verificar a disponibilidade de um alojamento, utiliza-se a técnica de \textit{pesquisa binária}, que permite encontrar rapidamente se há sobreposição de datas entre reservas. Abaixo está um exemplo de como o sistema verifica se existe disponibilidade:

\vspace*{15pt}
\renewcommand{\lstlistingname}{Amostra de Código}
\begin{lstlisting}[ 
	language=csh, 
	caption={Uso de Pesquisa Binária}, 
	label={lst:C}]
public bool IsAvailable(DateTime startDate, DateTime endDate)
{
	if (endDate <= startDate)
	throw new ArgumentException("End date must be after the start date.");
	
	int index = _reservedDates.BinarySearch((startDate, endDate), new DateRangeComparer());
	if (index < 0)
	index = ~index; // A pesquisa binária retorna o complemento do índice de inserção
	
	if (index > 0 && _reservedDates[index - 1].End > startDate)
	{
		return false; // Sobreposição com a reserva anterior
	}
	if (index < _reservedDates.Count && _reservedDates[index].Start < endDate)
	{
		return false; // Sobreposição com a reserva seguinte
	}
	
	return true; // Não há sobreposição, o alojamento está disponível
}

return true; // Não há sobreposição, o alojamento está disponível
\end{lstlisting}

Ao utilizar a \textit{List} de tuplas para armazenar intervalos de datas e a pesquisa binária para verificar a disponibilidade, o sistema é capaz de realizar verificações de disponibilidade de maneira muito mais eficiente do que uma busca linear. A \textit{pesquisa binária} reduz o tempo de busca de $O(n)$ para $O(\textit{log}(n))$, o que é um ganho significativo em termos de desempenho, principalmente em sistemas com muitas reservas.

\subsection{Vantagens das Estruturas de Dados Escolhidas}

As escolhas de \textit{Dictionary} e \textit{List} oferecem várias vantagens para o sistema:

\begin{itemize}
	\item \textbf{Eficiência de Acesso}: O \textit{Dictionary} permite acesso rápido às entidades, como clientes e reservas, através de suas chaves únicas. Isto melhora a performance, especialmente quando o sistema lida com um grande volume de dados.
	\item \textbf{Organização e Escalabilidade}: O uso do \textit{Dictionary} para armazenar entidades e o \textit{List} para reservas garante que o sistema seja facilmente escalável. À medida que o número de clientes ou reservas cresce, o desempenho do sistema permanece estável devido à eficiência destas estruturas.
	\item \textbf{Pesquisa Rápida de Intervalos de Datas}: A combinação de \textit{List} com pesquisa binária permite que a verificação de disponibilidade de um alojamento seja realizada de forma muito mais eficiente do que outras abordagens. A inserção das reservas na lista ordenada mantém a integridade dos dados e facilita a verificação de conflitos de datas.
\end{itemize}

Estas escolhas refletem uma consideração cuidadosa da necessidade de desempenho e eficiência, ao mesmo tempo que garantem uma implementação simples e direta, essencial para manter o sistema modular e fácil de manter.

\section{Layout do Sistema}

A arquitetura do sistema foi estruturada de acordo com um layout comum e amplamente utilizado em aplicações de software, que segue o padrão de separação de responsabilidades entre diferentes camadas. Esse padrão facilita a manutenção, escalabilidade e testabilidade do sistema. A divisão em \textit{Models}, \textit{Repositories}, \textit{Services} e \textit{Utilities} organiza a aplicação de maneira modular e coesa, garantindo que cada componente tenha uma função clara e bem definida.

\subsection{Models} A camada de \textit{Models} é responsável por representar os dados essenciais do sistema. Nesta camada, as classes de modelo (como \textit{Client}, \textit{Accommodation} e \textit{Reservation}) são definidas com propriedades e comportamentos relacionados aos dados que representam. Esses modelos são frequentemente utilizados para mapear as entidades que serão manipuladas pela aplicação, sendo a base para as operações de armazenamento e recuperação de dados. Além disso, os modelos podem incluir validações de dados e lógica de negócios específica, facilitando a comunicação entre as diferentes camadas.

\subsection{Repositories} A camada de \textit{Repositories} atua como a interface entre o sistema e a base de dados ou outro armazenamento persistente. Os repositórios são responsáveis por realizar as operações de leitura e escrita nos dados. Esta camada encapsula as complexidades do acesso aos dados e oferece métodos que permitem que a camada de serviço trabalhe com abstrações em vez de se preocupar diretamente com a persistência. No nosso sistema, por exemplo, o repositório de clientes pode fornecer métodos como \textit{Add()}, \textit{Remove()} e \textit{GetById()}, permitindo que os dados dos clientes sejam manipulados de forma segura e eficiente.

\subsection{Services} A camada de \textit{Services} contém a lógica de negócio principal da aplicação. Os serviços orquestram as interações entre os modelos e os repositórios, executando operações mais complexas que envolvem várias entidades ou ações. No contexto do nosso sistema de gestão de reservas, o \textit{BookingManager} pode ser considerado um serviço, pois ele gerencia o processo de reserva, incluindo a criação, validação e processamento de pagamentos. Os serviços permitem que a lógica de negócios seja centralizada e desacoplada das camadas de persistência e apresentação, tornando o sistema mais fácil de testar e de manter.

\subsection{Utilities} A camada de \textit{Utilities} contém classes e métodos auxiliares que oferecem funcionalidades que podem ser usadas em diferentes partes do sistema. Exemplos típicos incluem ferramentas de validação, manipulação de strings, loggers e geradores de relatórios. As \textit{Utilities} ajudam a manter o código limpo e reutilizável, evitando a duplicação de lógica em várias partes da aplicação. Por exemplo, um validador de datas ou um utilitário de formatação de preços pode ser usado em diferentes componentes, como na criação de reservas e no processamento de pagamentos.

\section{Conclusão}
\label{sec:conclusion}

w

%%% ===============================================================================
%%% Bibliography
%%% ===============================================================================

\renewcommand{\bibsection}{\section*{Referências}} % requried for natbib to have "References" printed and as section*, not chapter*
% Use natbib compatbile splncs04nat style.
% It does provide all features of splncs04.bst, but is developed in a clean way.
% Source: https://github.com/tpavlic/splncs04nat
\bibliographystyle{splncs04nat}
\begingroup
  \microtypecontext{expansion=sloppy}
  \small % ensure correct font size for the bibliography
  \bibliography{paper}
\endgroup

% Enfore empty line after bibliography
\ \\
%
Todas as ligações foram consultadas pela última vez no dia 13 de novembro de 2024.


%%% ===============================================================================
%\appendix
%\addcontentsline{toc}{chapter}{APPENDICES}

%\listoffigures
%\listoftables
%%% ===============================================================================

%%% ===============================================================================
%\section{My first appendix}\label{sec:appendix1}
%%% ===============================================================================
\end{document}
